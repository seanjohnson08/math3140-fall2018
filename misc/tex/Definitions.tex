Sure! Here are the answers to the homework questions:

\begin{itemize}
\item \href{http://en.wikipedia.org/wiki/Cartesian_product}{Cartesian product}: The Cartesian product of two sets $A$ and $B$, denoted $A \times B$, is the set of all ordered pairs $(a, b)$ where $a$ is an element of $A$ and $b$ is an element of $B$.
\item \href{http://en.wikipedia.org/wiki/Direct_product}{\defn{direct product}} and \defn{direct power}: The direct product of a family of sets is the set of all possible tuples where the $i$th element of the tuple is an element of the $i$th set in the family. The direct power of a set $A$ is the direct product of $A$ with itself a certain number of times.
\item \href{http://en.wikipedia.org/wiki/Finitary_relation}{relation}: A relation is a set of ordered pairs. It can be thought of as a set of connections between elements of two sets.
\item \href{http://en.wikipedia.org/wiki/Function_(mathematics)}{function}: A function is a relation between two sets where each element of the first set is connected to exactly one element of the second set.
\item \href{http://en.wikipedia.org/wiki/Operation_(mathematics)}{operation} and \href{http://en.wikipedia.org/wiki/Finitary}{finitary operation}: An operation is a function that takes one or more inputs and produces an output. A finitary operation is an operation that takes a finite number of inputs.
\item \href{http://en.wikipedia.org/wiki/Structure_(mathematical_logic)#Domain}{universe} or domain: The universe or domain of a structure is the set of all elements that the structure is defined over.
\item \href{http://en.wikipedia.org/wiki/Arity}{arity} of relation, function, or operation: The arity of a relation, function, or operation is the number of inputs it takes. For example, a binary relation takes two inputs, a unary function takes one input, and a ternary operation takes three inputs.
\item $n$-ary relation on a set $X$ (notation: $\rho \subseteq X^n$): An $n$-ary relation on a set $X$ is a subset of the Cartesian product $X^n$, where $n$ is the arity of the relation.
\item $n$-ary function from set $X$ to set $Y$ (notation: $f: X^n \rightarrow Y$): An $n$-ary function from set $X$ to set $Y$ is a function that takes $n$ inputs from $X$ and produces an output in $Y$.
\item $n$-ary operation on a set $X$ (notation: $f: X^n \rightarrow X$): An $n$-ary operation on a set $X$ is a function that takes $n$ inputs from $X$ and produces an output in $X$.
\item \href{http://en.wikipedia.org/wiki/Binary_relation#Relations_over_a_set}{properties} binary relations might satisfy: 
  \begin{itemize}
  \item \boldemph{reflexive}: A binary relation is reflexive if every element is related to itself.
  \item \boldemph{(anti)symmetric}: A binary relation is symmetric if for every pair of elements $(a, b)$