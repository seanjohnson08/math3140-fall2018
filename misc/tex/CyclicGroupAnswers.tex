\documentclass[12pt]{article}
\usepackage{graphics}
\usepackage{amsmath}
\usepackage{amssymb}
\usepackage{amsthm}
\usepackage{epsfig}
\usepackage{geometry}
\usepackage{fancyhdr}
\usepackage{graphpap}
% \usepackage{pstricks}
% \usepackage{pst-node}
\usepackage{scalefnt}
\usepackage{tikz}
\newcommand{\order}[1]{\ensuremath{|#1|}}
\newcommand{\units}[1]{\ensuremath{U(#1)}}
\newcommand{\divides}{\ensuremath{\mid}}
\newcommand{\<}{\ensuremath{\langle}}
\renewcommand{\>}{\ensuremath{\rangle}}
\newcommand{\eye}{\ensuremath{e}}

\pagestyle{fancy} \lhead{\bf } \chead{\bf Cyclic Group Exercises: Answers}
\rhead{} \lfoot{} \cfoot{\thepage} \rfoot{}
\renewcommand{\headrulewidth}{0.6pt}
\renewcommand{\footrulewidth}{0.6pt}
\setlength{\headwidth}{6.5in}
% Fuzz -------------------------------------------------------------------
\hfuzz2pt % Don't bother to report over-full boxes if over-edge is < 2pt
% Line spacing -----------------------------------------------------------
\newlength{\defbaselineskip}
\setlength{\defbaselineskip}{\baselineskip}
\newcommand{\setlinespacing}[1]
           {\setlength{\baselineskip}{#1 \defbaselineskip}}
\newcommand{\doublespacing}{\setlength{\baselineskip}%
                           {2.0 \defbaselineskip}}
\newcommand{\singlespacing}{\setlength{\baselineskip}{\defbaselineskip}}
\setlength{\textwidth}{6.5in} \setlength{\textheight}{9in}
%\setlength{\parindent}{0mm}
\setlength{\oddsidemargin}{.1in}
\setlength{\evensidemargin}{.1in} \setlength{\voffset}{-.5in}
\setlength{\topmargin}{0pt}
% MATH -------------------------------------------------------------------
\newcommand{\A}{{\cal A}}
\newcommand{\h}{{\cal H}}
\newcommand{\s}{{\cal S}}
\newcommand{\W}{{\cal W}}
\newcommand{\D}{\textbf{D}}
\newcommand{\BH}{\mathbf B(\cal H)}
\newcommand{\KH}{\cal  K(\cal H)}
\newcommand{\Real}{\mathbb R}
\newcommand{\R}{\mathbb R}
\newcommand{\C} [1]{{\mathcal #1}}
\newcommand{\B} [1] {{\mathbf #1}}
\newcommand{\Q }{{\mathbb Q}}
\newcommand{\cm} {{\mathbb C}}
\newcommand{\Z} {{\mathbb Z}}
\newcommand{\PP} {{\mathbb P}}
\newcommand{\N }{{\mathbb N}}
\newcommand{\f} {{\mathbb F}}
\newcommand{\gl}{\mathop{\rm GL} }
\newcommand{\ord}{\mathop{\rm ord} }
\newcommand{\lcm}{\mathop{\rm lcm} }
\newcommand{\id}{\mathop{\rm id} }
\newcommand{\GL}{\mathop{\rm GL} }
\newcommand{\Ker}{\mathop{\rm Ker} }
\newcommand{\Adj}{\mathop{\rm Adj} }
\newcommand{\Imm}{\mathop{\rm Im} }
\newcommand{\RR}{\mathbb R}
\newcommand{\NN}{\mathbb N}
\newcommand{\CC}{\mathbb C}
\newcommand{\Complex}{\mathbb C}
\newcommand{\Field}{\mathbb F}
\newcommand{\RPlus}{[0,\infty)}
\newcommand{\norm}[1]{\left\Vert#1\right\Vert}
\newcommand{\essnorm}[1]{\norm{#1}_{\text{\rm\normalshape ess}}}
\newcommand{\abs}[1]{\left\vert#1\right\vert}
\newcommand{\set}[1]{\left\{#1\right\}}
\newcommand{\seq}[1]{\left<#1\right>}
\newcommand{\eps}{\varepsilon}
\newcommand{\To}{\longrightarrow}
\newcommand{\RE}{\operatorname{Re}}
\newcommand{\IM}{\operatorname{Im}}
\newcommand{\Poly}{{\cal{P}}(E)}
\newcommand{\EssD}{{\cal{D}}}
% THEOREMS ---------------------------------------------------------------
\theoremstyle{plain}
\newtheorem{thm}{Theorem}%[section]
\newtheorem{cor}[thm]{Corollary}
\newtheorem{lem}[thm]{Lemma}
\newtheorem{prop}[thm]{Proposition}
%
\theoremstyle{definition}
\newtheorem{defn}[thm]{Definition}
%
%\theoremstyle{remark}
\newtheorem{rem}[thm]{Remark}
\newtheorem{rems}[thm]{Remarks}
\newtheorem{ex}[thm]{Example}
\newtheorem{exs}[thm]{Examples}
\begin{document}

\begin{enumerate}
\item \begin{enumerate}
\item Generators are $g^k$ for $1\le k\le 4$.
\item Generators are $g^k$ for $k\in\set{1,\,3,\,7,\,9}$.
\item Generators are $g^{2k-1}$ for $1\le k\le 8$.
\item Generators are $g^k$ for $k\in\set{1,\,3,\,7,\,9,\,11,\,13,\,17,\,19}$.
\end{enumerate}
\item \begin{enumerate}
\item Generators of $\Z_5$ are $k$ for $1\le k\le 4$.
\item Generators of $\Z_{10}$ are $k$ for $k\in\set{1,\,3,\,7,\,9}$.
\item Generators of $\Z_{16}$ are $2k-1$ for $1\le k\le 8$.
\item Generators of $\Z_{20}$ are $k$ for $k\in\set{1,\,3,\,7,\,9,\,11,\,13,\,17,\,19}$.
\end{enumerate}
\item \begin{enumerate}
\item $\units{7}$ is cyclic with generator 3.
\item $\units{12}$ is not cyclic: every nonidentity element has order 2, but
$\units{12}$ has order 8.
\item $\units{16}$ is not cyclic: the nonidentity elements have orders 2 or 4, but $\units{16}$ has order 8.
\item $\units{11}$ is cyclic with generator 2.
\end{enumerate}
\item
(a) $\order{g^2}=10$\qquad (b) $\order{g^8}=5$ \qquad (c) $\order{g^5}=4$\qquad (d)
$\order{g^3}=20$
\item \begin{enumerate}
\item Subgroups:  $H_1=\langle e\rangle$, $H_2=\langle g^2\rangle$,
$H_3=\langle g^4\rangle$, $H_4=G$.
\item Subgroups: $H_1=\langle e\rangle$, $H_2=\langle g^2\rangle$,
$H_3=\langle g^5\rangle$, $H_4=G$.
\item Subgroups: $H_1=\langle e\rangle$, $H_2=\langle g^2\rangle$,
$H_3=\langle g^3\rangle$, $H_4=\langle g^6\rangle$, 
$H_5=\langle g^9\rangle $, $H_6=G$.
\item Subgroups $H_1=\langle e\rangle$, $H_2=\langle g^p\rangle$,
$H_3=\langle g^{p^2}\rangle$, $H_4=G$.
\item Subgroups $H_1=\langle e\rangle$, $H_2=\langle g^p\rangle$,
$H_3=\langle g^{q}\rangle$, $H_4=G$.
\item Subgroups 
$H_1=\langle e\rangle$, $H_2=\langle g^p\rangle$,
$H_3=\langle g^{p^2}\rangle$, $H_4=\langle g^q\rangle$, 
$H_5=\langle g^{pq}\rangle$,  $H_6=G$.
\end{enumerate}
%\item Done in class.
\item \label{item:1}
\begin{enumerate}
\item $H=\langle a\rangle$
\item $H=\langle a^2\rangle$
\item $H=\langle a^d\rangle$
\item $H=G$
\end{enumerate}
\end{enumerate}

\newpage

\pagestyle{fancy} \lhead{\bf } \chead{\bf Cyclic Group Exercises: Solutions}
\rhead{} \lfoot{} \cfoot{\thepage} \rfoot{}

\noindent Below are detailed solutions to a couple of the exercises.

\medskip

\noindent {\bf Exercise 6.} Part (c)\\
\\
{\bf Claim:}
If $G = \<a\>$ and $x = x^m$, $y = a^k$, then the subgroup, 
generated by $x$ and $y$, is 
$H = \<x, y\> = \<a^d\>$, where $d = \gcd(m, k)$.
\begin{proof}
If $d = \gcd(m, k)$, then there exist integers $r, s$ such that 
$d = rm + sk$.  Therefore, 
$a^d = a^{rm + sk} = a^{rm + sk} = a^{rm}a^{sk} =  x^ry^s$. This proves that
$a^d \in \<x, y\>$, so 
$\<a^d\> \subseteq \<x, y\>$. On the other hand, 
$d\divides m$, so
$m = \alpha d$ and $x = a^m = a^{\alpha d} = (a^d)^\alpha$, so 
$x\in \<a^d\>$.  
Similarly, $d\divides k$, so $k = \beta d$ 
and $y = a^k = a^{\beta d} = (a^d)^\beta$, so
$y\in \<a^d\>$.  Therefore, $\<x, y\> \subseteq \<a^d\>$.
\end{proof}

\bigskip

\noindent {\bf Exercise 5.} Part (f)\\
\\
If $\order{g}=p^2q$, then the subgroups of $G= \<g\>$ are
\[
G, \quad
\<g^p\>, \quad 
\<g^{p^2}\>, \quad 
\<g^q\>, \quad 
\<g^{pq}\>, \quad 
\<g^{p^2q}\> = \<e\>,
\]
and we have 
$\<g^m\> \leq \< g^k\>$ if and only if $k\divides m$.  Therefore, the subgroup
lattice is given by the Hasse diagram below.

\begin{center}
\begin{tikzpicture}[scale=1]
  \node (1) at (0,0) {$\langle \eye \rangle$}; 
  \node (6) at (2,2) {$\langle g^{pq} \rangle$}; 
  \node (4) at (-2,2) {$\langle g^{p^2} \rangle$}; 
  \node (2) at (0,4) {$\langle g^p \rangle$}; 
  \node (3) at (4,4) {$\langle g^q \rangle$}; 
  \node (G) at (2,6) {$\langle g \rangle$}; 

  \draw (1) to (6) to (2) to (4) to (1);
  \draw (6) to (3) to (G) to (2) to (4);

\end{tikzpicture}
\end{center}

\end{document}
