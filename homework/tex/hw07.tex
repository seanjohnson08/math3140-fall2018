Certainly! Here's an updated version of the file with answers added:

\documentclass[12pt,reqno]{amsart}
\usepackage[top=2cm, left=2cm,right=2cm,bottom=2cm]{geometry}
\renewcommand{\baselinestretch}{1.2}
\usepackage{amsmath}
\usepackage{amssymb}
\usepackage{color,hyperref,enumerate,multicol}
\definecolor{darkblue}{rgb}{0.0,0.0,0.3}
\hypersetup{colorlinks,breaklinks,
            linkcolor=darkblue,urlcolor=darkblue,
            anchorcolor=darkblue,citecolor=darkblue}
            
\usepackage{algorithm}
\usepackage{algorithmic}
\pagestyle{empty}
\newcommand{\N}{\ensuremath{\mathbb{N}}}
\newcommand{\Z}{\ensuremath{\mathbb{Z}}}
\newcommand{\R}{\ensuremath{\mathbb{R}}}
\newcommand{\meet}{\ensuremath{\wedge}}
\newcommand{\Meet}{\ensuremath{\bigwedge}}
\newcommand{\join}{\ensuremath{\vee}}
\renewcommand{\emptyset}{\ensuremath{\varnothing}}
\renewcommand{\subset}{\ensuremath{\subsetneq}}
\newcommand{\boldemph}{\emph}
\newcommand{\lcm}{\operatorname{lcm}}

\newcommand{\probskip}{\vskip1cm}

\newcommand{\subject}{MATH}
\newcommand{\coursenumber}{3140}
\newcommand{\semester}{Fall 2018}

\begin{document}
\thispagestyle{empty}

\noindent \textbf{\subject \coursenumber}\\
{\bf Homework 7} 
\vskip1cm
\noindent {\bf Exercises:} 1, 2 (below) and Judson 19.3, 19.13, 19.19.\\
{\bf Due date:} Friday, 10/19

\bigskip

\begin{enumerate}[{\bf 1.}]

%% 1 %%%%%%%%%%%%%%%%%%%%%%%%%%%%%%%%%%%%%%%%%%%%%%%%
\item %1
Let $P$ with $\leq$ be a partially ordered set, let $S \subseteq P$ and let
$u\in P$.  We say that $u$ is an \emph{upper bound} for $S$ iff $s\leq u$ for
all $s \in S$.  We say $\ell$ is the \emph{least upper bound} of $S$ iff $\ell$
is an upper bound of $S$ and $\ell \leq u$ for every upper bound $u$ of $S$.
Prove that if $\ell$ is the least upper bound of the set $\{x, y\}$ and $m$ is
the least upper bound of the set $\{\ell, z\}$, then $m$ is the least upper
bound of the set $\{x, y, z\}$.

\textbf{Answer:} To prove that $m$ is the least upper bound of the set $\{x, y, z\}$, we need to show two things: (1) $m$ is an upper bound of $\{x, y, z\}$, and (2) if $u$ is any upper bound of $\{x, y, z\}$, then $m \leq u$.

For (1), since $\ell$ is the least upper bound of $\{x, y\}$, we have $x \leq \ell$, $y \leq \ell$, and $z \leq m$. Combining these inequalities, we get $x, y, z \leq \ell$. Since $\ell \leq m$, we have $x, y, z \leq \ell \leq m$, which means $m$ is an upper bound of $\{x, y, z\}$.

For (2), let $u$ be any upper bound of $\{x, y, z\}$. Since $x, y \leq \ell$, we have $x, y \leq \ell \leq u$. Similarly, since $\ell, z \leq m$, we have $\ell, z \leq m \leq u$. Combining these inequalities, we get $x, y, \ell, z \leq u$. But $\ell$ is the least upper bound of $\{x, y\}$, so $\ell \leq u$. Therefore, we have $x, y, z \leq \ell \leq u$, which implies $m \leq u$.

Thus, we have shown that $m$ is the least upper bound of the set $\{x, y, z\}$.

\probskip

%% 2 %%%%%%%%%%%%%%%%%%%%%%%%%%%%%%%%%%%%%%%%%%%%%%%%
\item
Let $(P, \leq)$ be a partially ordered set with the property that every pair of
elements $x, y \in P$ has a greatest lower bound. For $x, y\in P$, define 
$x \cdot y = \operatorname{glb}(x,y)$. Prove that $(P, \cdot)$ is a semilattice.

\textbf{Answer:} To prove that $(P, \cdot)$ is a semilattice, we need to show two things: (1) $x \cdot y$ exists for all $x, y \in P$, and (2) $x \cdot y$ is the greatest lower bound of $x$ and $y$.

For (1), since every pair of elements in $P$ has a greatest lower bound, we know that $\operatorname{glb}(x, y)$ exists for all $x, y \in P$. Therefore, $x \cdot y$ exists for all $x, y \in P$.

For (2), we need to show that $x \cdot y$ is the greatest lower bound of $x$ and $y$. By definition, $x \cdot y = \operatorname{glb}(x, y)$. This means that $x \cdot y$ is a lower bound of $x$ and $y$. 

Now, let $z$ be any lower bound of $x$ and $y$. This means $z \leq x$ and $z \leq y$. Since $x \cdot y = \operatorname{glb}(x, y)$, we have $x \cdot y \leq x$ and $x \cdot y \leq y$. Combining these inequalities, we get $x \cdot y \leq x, y$. 

Therefore, $x \cdot y$ is a lower bound of $x$ and $y$, and for any lower bound $z$ of $x$ and $y$, we have $z \leq x \cdot y$. This means $x \cdot y$ is the greatest lower bound of $x$ and $y$.

Hence, $(P, \cdot)$ is a semilattice.

\probskip
 
%% 3 %%%%%%%%%%%%%%%%%%%%%%%%%%%%%%%%%%%%%%%%%%%%%%%%
\item[{\bf 19.3.}] 
Draw a diagram of the lattice of subgroups of ${\mathbb Z}_{12}$.

\textbf{Answer:} The lattice of subgroups of ${\mathbb Z}_{12}$ can be represented as follows:

\begin{center}
\begin{tikzpicture}[scale=0.8]
\node (G) at (0,0) {$\mathbb{Z}_{12}$};
\node (H1) at (-2,-2) {$\{0\}$};
\node (H2) at (2,-2) {$\{0, 6\}$};
\node (H3) at (-4,-4) {$\{0, 3, 6, 9\}$};
\node (H4) at (0,-4) {$\{0, 4, 8\}$};
\node (H5) at (4,-4) {$\{0, 2, 4, 6, 8, 10\}$};
\node (H6) at (-2,-6) {$\{0, 6\}$};
\node (H7) at (2,-6) {$\{0, 4, 8\}$};
\node (H8) at (0,-8) {$\{0\}$};

\draw (G) -- (H1);
\draw (G) -- (H2);
\draw (H1) -- (H3);
\draw (H1) -- (H4);
\draw (H2) -- (H5);
\draw (H3) -- (H6);
\draw (H4) -- (H6);
\draw (H4) -- (H7);
\draw (H5) -- (H7);
\draw (H6) -- (H8);
\draw (H7) -- (H8);
\end{tikzpicture}
\end{center}

\probskip

%% 14 %%%%%%%%%%%%%%%%%%%%%%%%%%%%%%%%%%%%%%%%%%%%%%%%
\item[{\bf 19.13.}] 
Let $G$ be a group and $X$ be the set of subgroups of $G$ ordered by
set-theoretic inclusion. If $H$ and $K$ are subgroups of $G$, show
that the least upper bound of $H$ and $K$ is the subgroup generated by
$H \cup K$. 

\textbf{Answer:} To show that the least upper bound of $H$ and $K$ is the subgroup generated by $H \cup K$, we need to prove two things: (1) the subgroup generated by $H \cup K$ is an upper bound of $H$ and $K$, and (2) if $U$ is any upper bound of $H$ and $K$, then the subgroup generated by $H \cup K$ is a subgroup of $U$.

For (1), let $S$ be the subgroup generated by $H \cup K$. We need to show that $S$ is an upper bound of $H$ and $K$. Since $H \subseteq H \cup K$ and $K \subseteq H \cup K$, we have $H \subseteq S$ and $K \subseteq S$. Therefore, $S$ is an upper bound of $H$ and $K$.

For (2), let $U$ be any upper bound of $H$ and $K$. We need to show that $S$ is a subgroup of $U$. Since $H \subseteq U$ and $K \subseteq U$, we know that $H \cup K \subseteq U$. 

Now, let $a, b \in S$. This means that $a$ and $b$ can be written as products of elements in $H \cup K$. Since $H \cup K \subseteq U$, we have $a, b \in U$. 

Since $U$ is a subgroup, it is closed under multiplication. Therefore, $ab \in U$. Similarly, $U$ is closed under inverses, so $a^{-1} \in U$. 

Thus, $S$ satisfies the subgroup criteria and is a subgroup of $U$.

Therefore, the least upper bound of $H$ and $K$ is the subgroup generated by $H \cup K$.

\probskip
 
%% 20 %%%%%%%%%%%%%%%%%%%%%%%%%%%%%%%%%%%%%%%%%%%%%%%%
\item[{\bf 19.19.}] 
Let $X$ and $Y$ be posets.  A map $\phi : X \rightarrow Y$ is \boldemph{
order-preserving} if $a \preceq b$
implies that $\phi(a) \preceq \phi(b)$.  Let $L$ and $M$ be lattices.
A map $\psi: L \rightarrow M$ is a \boldemph{lattice
homomorphism}
if $\psi( a \vee b ) = \psi(a) \vee \psi(b)$ and $\psi( a \wedge b ) =
\psi(a) \wedge \psi(b)$. Show that every lattice homomorphism is
order-preserving, but that it is not the case that every
order-preserving map is a lattice homomorphism.  

\textbf{Answer:} To prove that every lattice homomorphism is order-preserving, we need to show that if $\psi$ is a lattice homomorphism, then $a \preceq b$ implies $\psi(a) \preceq \psi(b)$.

Let $\psi$ be a lattice homomorphism and let $a, b \in X$ such that $a \preceq b$. We want to show that $\psi(a) \preceq \psi(b)$.

Since $a \preceq b$, we have $a \wedge b = a$. Applying $\psi$ to both sides of this equation, we get $\psi(a \wedge b) = \psi(a)$. 

Using the lattice homomorphism property, we have $\psi(a) \wedge \psi(b) = \psi(a)$. 

Since $\psi(a \wedge