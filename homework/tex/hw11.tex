Certainly! Here's the updated file with answers added inline:

```tex
\documentclass[12pt,reqno]{amsart}
\usepackage[top=2cm, left=2cm,right=2cm,bottom=2cm]{geometry}
\renewcommand{\baselinestretch}{1.2}
\usepackage{amsmath}
\usepackage{amssymb}
\usepackage{scalefnt}
\usepackage{tikz}
\usepackage{color,hyperref,enumerate,multicol}
\definecolor{darkblue}{rgb}{0.0,0.0,0.3}
\hypersetup{colorlinks,breaklinks,
            linkcolor=darkblue,urlcolor=darkblue,
            anchorcolor=darkblue,citecolor=darkblue}
            
\usepackage{algorithm}
\usepackage{algorithmic}
\pagestyle{empty}
\newcommand{\N}{\ensuremath{\mathbb{N}}}
\newcommand{\Z}{\ensuremath{\mathbb{Z}}}
\newcommand{\R}{\ensuremath{\mathbb{R}}}
\newcommand{\bL}{\ensuremath{\mathbf{L}}}
\newcommand{\bP}{\ensuremath{\mathbf{P}}}
\newcommand{\bQ}{\ensuremath{\mathbf{Q}}}
\newcommand{\bA}{\ensuremath{\mathbf{A}}}
\newcommand{\bB}{\ensuremath{\mathbf{B}}}
\newcommand{\bG}{\ensuremath{\mathbf{G}}}
\newcommand{\bH}{\ensuremath{\mathbf{H}}}
\newcommand{\invG}{\ensuremath{\operatorname{inv}^{\bG}}}
\newcommand{\invH}{\ensuremath{\operatorname{inv}^{\bH}}}
\newcommand{\meet}{\ensuremath{\wedge}}
\newcommand{\Meet}{\ensuremath{\bigwedge}}
\newcommand{\<}{\ensuremath{\langle}}
\renewcommand{\>}{\ensuremath{\rangle}}
\newcommand{\join}{\ensuremath{\vee}}
\renewcommand{\emptyset}{\ensuremath{\varnothing}}
\renewcommand{\subset}{\ensuremath{\subsetneq}}
\newcommand{\boldemph}{\emph}
\newcommand{\lcm}{\ensuremath{\operatorname{lcm}}}
\newcommand{\Sym}{\ensuremath{\operatorname{Sym}}}
%\newcommand{\bG}{\ensuremath{\mathbf{G}}}

\newcommand{\probskip}{\vskip5mm}
\usepackage{xspace}
\newcommand{\subject}{MATH\xspace}
\newcommand{\coursenumber}{3140\xspace}
\newcommand{\semester}{Fall 2018\xspace}
\newcommand{\exercises}{Ch.~11. 7, 11, 16, 17, 18.}
\newcommand{\due}[1]{{\bf Due:} #1}
\newcommand{\hwheading}{\textbf{\subject \coursenumber -- Homework 11 \\ \due{Monday, 26 November 2018}}}
\begin{document}
\thispagestyle{empty}

\noindent \hwheading

\noindent {\bf Exercises:} \exercises 

\bigskip

\begin{enumerate}
%% 1 %%%%%%%%%%%%%%%%%%%%%%%%%%%%%%%%%%%%%%%%%%%%%%%%
\item[{\bf 11.7}] 
In the group ${\mathbb Z}_{24}$, let $H = \langle 4 \rangle$ and $N =
\langle 6 \rangle$. 
\begin{enumerate}
 
 \item
List the elements in $HN$ (we usually write $H + N$ for these additive
groups) and $H \cap N$. 

\textbf{Answer:} The elements in $HN$ are $\{0, 4, 8, 12, 16, 20\}$ and the elements in $H \cap N$ are $\{0, 12\}$.
 
 \item
List the cosets in $HN/N$, showing the elements in each coset.

\textbf{Answer:} The cosets in $HN/N$ are $\{0 + N, 4 + N\}$ and $\{8 + N, 12 + N\}$.

 
 \item
List the cosets in $H/(H \cap N)$, showing the elements in each coset. 

\textbf{Answer:} The cosets in $H/(H \cap N)$ are $\{0 + (H \cap N), 4 + (H \cap N)\}$ and $\{8 + (H \cap N), 16 + (H \cap N)\}$.

 
 \item
Give the correspondence between $HN/N$ and $H/(H \cap N)$ described in
the proof of the Second Isomorphism Theorem. 

\textbf{Answer:} The correspondence between $HN/N$ and $H/(H \cap N)$ is given by the map $\phi: HN/N \to H/(H \cap N)$ defined by $\phi(a + N) = a + (H \cap N)$.

\end{enumerate}


\bigskip

\item[{\bf 11.11}]
Show that a homomorphism defined on a cyclic group is completely
determined by its action on the generator of the group.

\textbf{Answer:} Let $G$ be a cyclic group generated by $g$. Suppose $\phi: G \to H$ is a homomorphism. Then for any $x \in G$, we can write $x = g^n$ for some integer $n$. Since $\phi$ is a homomorphism, we have $\phi(x) = \phi(g^n) = \phi(g)^n$. Therefore, the action of $\phi$ on the generator $g$ completely determines the action of $\phi$ on all elements of $G$.

\bigskip

\item[{\bf 11.16}]
If $H$ and $K$ are normal subgroups of $G$ and $H \cap K = \{ e \}$,
prove that $G$ is isomorphic to a subgroup of $G/H \times G/K$.

\textbf{Answer:} Let $\phi: G \to G/H \times G/K$ be defined by $\phi(g) = (gH, gK)$. We claim that $\phi$ is a homomorphism and an isomorphism.

To show that $\phi$ is a homomorphism, let $g_1, g_2 \in G$. Then
\begin{align*}
    \phi(g_1g_2) &= (g_1g_2H, g_1g_2K) \\
    &= (g_1Hg_2H, g_1Kg_2K) \quad \text{(since $H$ and $K$ are normal)} \\
    &= (g_1H, g_1K)(g_2H, g_2K) \\
    &= \phi(g_1)\phi(g_2).
\end{align*}
Therefore, $\phi$ is a homomorphism.

To show that $\phi$ is injective, suppose $\phi(g_1) = \phi(g_2)$. Then $(g_1H, g_1K) = (g_2H, g_2K)$, which implies $g_1H = g_2H$ and $g_1K = g_2K$. Since $H \cap K = \{e\}$, we have $g_1 = g_2$. Therefore, $\phi$ is injective.

To show that $\phi$ is surjective, let $(H', K') \in G/H \times G/K$. Since $H$ and $K$ are normal subgroups, we can choose $g_1 \in H'$ and $g_2 \in K'$. Then $(g_1, g_2) \in G$ and $\phi(g_1g_2) = (g_1H, g_1K) = (H', K')$. Therefore, $\phi$ is surjective.

Since $\phi$ is a homomorphism and an isomorphism, it follows that $G$ is isomorphic to a subgroup of $G/H \times G/K$.

\bigskip

\newcommand\GHGK{\ensuremath{G/H \times G/K}}
\item[{\bf 11.17}]
Let $\varphi : G_1 \rightarrow G_2$ be a surjective group homomorphism.
Let $H_1$ be a normal subgroup of $G_1$ and suppose that $\varphi(H_1) =
H_2$.  Prove or disprove that $G_1/H_1 \cong G_2/H_2$.

\textbf{Answer:} The statement is true. Let $\psi: G_1/H_1 \to G_2/H_2$ be defined by $\psi(gH_1) = \varphi(g)H_2$. We claim that $\psi$ is a well-defined isomorphism.

To show that $\psi$ is well-defined, suppose $g_1H_1 = g_2H_1$. Then $g_2^{-1}g_1 \in H_1$, and since $H_1$ is normal, we have $\varphi(g_2^{-1}g_1) \in H_2$. Therefore, $\varphi(g_2)^{-1}\varphi(g_1) \in H_2$, which implies $\varphi(g_1)H_2 = \varphi(g_2)H_2$. Hence, $\psi$ is well-defined.

To show that $\psi$ is an isomorphism, we need to show that it is a homomorphism, injective, and surjective.

To show that $\psi$ is a homomorphism, let $g_1, g_2 \in G_1$. Then
\begin{align*}
    \psi((g_1H_1)(g_2H_1)) &= \psi(g_1g_2H_1) \\
    &= \varphi(g_1g_2)H_2 \\
    &= (\varphi(g_1)\varphi(g_2))H_2 \\
    &= \varphi(g_1)H_2\varphi(g_2)H_2 \\
    &= \psi(g_1H_1)\psi(g_2H_1).
\end{align*}
Therefore, $\psi$ is a homomorphism.

To show that $\psi$ is injective, suppose $\psi(g_1H_1) = \psi(g_2H_1)$. Then $\varphi(g_1)H_2 = \varphi(g_2)H_2$, which implies $\varphi(g_1^{-1}g_2) \in H_2$. Since $\varphi$ is surjective, there exists $h \in H_1$ such that $\varphi(h) = \varphi(g_1^{-1}g_2)$. Therefore, $g_1^{-1}g_2h^{-1} \in H_1$, which implies $g_1H_1 = g_2H_1$. Hence, $\psi$ is injective.

To show that $\psi$ is surjective, let $gH_2 \in G_2/H_2$. Since $\varphi$ is surjective, there exists $h \in G_1$ such that $\varphi(h) = g$. Then $\psi(hH_1) = \varphi(h)H_2 = gH_2$. Therefore, $\psi$ is surjective.

Since $\psi$ is a well-defined isomorphism, it follows that $G_1/H_1 \cong G_2/H_2$.

\bigskip

\item[{\bf 11.18}]
Let $\phi : G \rightarrow H$ be a group homomorphism.  Show that
$\phi$ is one-to-one if and only if $\phi^{-1}(e) = \{ e \}$.

\textbf{Answer:} 

$(\Rightarrow)$ Suppose $\phi$ is one-to-one. Let $g \in \phi^{-1}(e)$. Then $\phi(g) = e$. Since $\phi$ is one-to-one, we have $g = e$. Therefore, $\phi^{-1}(e) = \{e\}$.

$(\Leftarrow)$ Suppose $\phi^{-1}(e) = \{e\}$. Let $g_1, g_2 \in G$ such that $\phi(g_1) = \phi(g_2)$. Then $\phi(g_1)\phi(g_2)^{-1} = e$, which implies $\phi(g_1g_2^{-1}) = e$. Therefore, $g_1g_2^{-1} \in \phi^{-1}(e