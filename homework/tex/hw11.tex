\documentclass[12pt,reqno]{amsart}
\usepackage[top=2cm, left=2cm,right=2cm,bottom=2cm]{geometry}
\renewcommand{\baselinestretch}{1.2}
\usepackage{amsmath}
\usepackage{amssymb}
\usepackage{scalefnt}
\usepackage{tikz}
\usepackage{color,hyperref,enumerate,multicol}
\definecolor{darkblue}{rgb}{0.0,0.0,0.3}
\hypersetup{colorlinks,breaklinks,
            linkcolor=darkblue,urlcolor=darkblue,
            anchorcolor=darkblue,citecolor=darkblue}
            
\usepackage{algorithm}
\usepackage{algorithmic}
\pagestyle{empty}
\newcommand{\N}{\ensuremath{\mathbb{N}}}
\newcommand{\Z}{\ensuremath{\mathbb{Z}}}
\newcommand{\R}{\ensuremath{\mathbb{R}}}
\newcommand{\bL}{\ensuremath{\mathbf{L}}}
\newcommand{\bP}{\ensuremath{\mathbf{P}}}
\newcommand{\bQ}{\ensuremath{\mathbf{Q}}}
\newcommand{\bA}{\ensuremath{\mathbf{A}}}
\newcommand{\bB}{\ensuremath{\mathbf{B}}}
\newcommand{\bG}{\ensuremath{\mathbf{G}}}
\newcommand{\bH}{\ensuremath{\mathbf{H}}}
\newcommand{\invG}{\ensuremath{\operatorname{inv}^{\bG}}}
\newcommand{\invH}{\ensuremath{\operatorname{inv}^{\bH}}}
\newcommand{\meet}{\ensuremath{\wedge}}
\newcommand{\Meet}{\ensuremath{\bigwedge}}
\newcommand{\<}{\ensuremath{\langle}}
\renewcommand{\>}{\ensuremath{\rangle}}
\newcommand{\join}{\ensuremath{\vee}}
\renewcommand{\emptyset}{\ensuremath{\varnothing}}
\renewcommand{\subset}{\ensuremath{\subsetneq}}
\newcommand{\boldemph}{\emph}
\newcommand{\lcm}{\ensuremath{\operatorname{lcm}}}
\newcommand{\Sym}{\ensuremath{\operatorname{Sym}}}
%\newcommand{\bG}{\ensuremath{\mathbf{G}}}

\newcommand{\probskip}{\vskip5mm}
\usepackage{xspace}
\newcommand{\subject}{MATH\xspace}
\newcommand{\coursenumber}{3140\xspace}
\newcommand{\semester}{Fall 2018\xspace}
\newcommand{\exercises}{Ch.~11. 7, 11, 17, 18, 19.}
\newcommand{\due}[1]{{\bf Due:} #1}
\newcommand{\hwheading}{\textbf{\subject \coursenumber -- Homework 11 \\ \due{Monday, 26 November 2018}}}
\begin{document}
\thispagestyle{empty}

\noindent \hwheading

\noindent {\bf Exercises:} \exercises 

\bigskip

\begin{enumerate}
%% 1 %%%%%%%%%%%%%%%%%%%%%%%%%%%%%%%%%%%%%%%%%%%%%%%%
\item[{\bf 11.7}] 
In the group ${\mathbb Z}_{24}$, let $H = \langle 4 \rangle$ and $N =
\langle 6 \rangle$. 
\begin{enumerate}
 
 \item
List the elements in $HN$ (we usually write $H + N$ for these additive
groups) and $H \cap N$. 
 
 \item
List the cosets in $HN/N$, showing the elements in each coset.
 
 \item
List the cosets in $H/(H \cap N)$, showing the elements in each coset. 
 
 \item
Give the correspondence between $HN/N$ and $H/(H \cap N)$ described in
the proof of the Second Isomorphism Theorem. 
\end{enumerate}


\bigskip

\item[{\bf 11.11}]
Show that a homomorphism defined on a cyclic group is completely
determined by its action on the generator of the group.

\bigskip

\item[{\bf 11.17}]
If $H$ and $K$ are normal subgroups of $G$ and $H \cap K = \{ e \}$,
prove that $G$ is isomorphic to a subgroup of $G/H \times G/K$.

\bigskip

\newcommand\GHGK{\ensuremath{G/H \times G/K}}
\item[{\bf 11.18}]
Let $\varphi : G_1 \rightarrow G_2$ be a surjective group homomorphism.
Let $H_1$ be a normal subgroup of $G_1$ and suppose that $\varphi(H_1) =
H_2$.  Prove or disprove that $G_1/H_1 \cong G_2/H_2$.
 
\bigskip

\item[{\bf 11.19}]
Let $\phi : G \rightarrow H$ be a group homomorphism.  Show that
$\phi$ is one-to-one if and only if $\phi^{-1}(e) = \{ e \}$.

\end{enumerate}


\end{document}


