Certainly! Here's an updated version of the file with answers added:

\documentclass[12pt,reqno]{amsart}
\usepackage[top=1.5cm, left=1.5cm,right=1.5cm,bottom=1.5cm]{geometry}
\renewcommand{\baselinestretch}{1.2}
\usepackage{amsmath}
\usepackage{amssymb}
\usepackage{color,hyperref,enumerate,multicol}
\definecolor{darkblue}{rgb}{0.0,0.0,0.3}
\hypersetup{colorlinks,breaklinks,
            linkcolor=darkblue,urlcolor=darkblue,
            anchorcolor=darkblue,citecolor=darkblue}
            
\usepackage{algorithm}
\usepackage{algorithmic}
\pagestyle{empty}
\newcommand{\N}{\ensuremath{\mathbb{N}}}
\newcommand{\Z}{\ensuremath{\mathbb{Z}}}
\newcommand{\R}{\ensuremath{\mathbb{R}}}
\newcommand{\meet}{\ensuremath{\wedge}}
\newcommand{\Meet}{\ensuremath{\bigwedge}}
\newcommand{\join}{\ensuremath{\vee}}
\renewcommand{\emptyset}{\ensuremath{\varnothing}}
\renewcommand{\subset}{\ensuremath{\subsetneq}}
\newcommand{\boldemph}{\emph}
\newcommand{\lcm}{\operatorname{lcm}}

\begin{document}
\thispagestyle{empty}

\noindent \textbf{Math 3140} \hfill {\bf Homework 3} 
\vskip1cm
\noindent {\bf Chapter 3:} 1bd, 2bd, 3, 5, 7, 12.  
\\
{\bf Due date:} Wednesday, 9/19

\medskip

\begin{enumerate}

%% 1 %%%%%%%%%%%%%%%%%%%%%%%%%%%%%%%%%%%%%%%%%%%%%%%%
\item[{\bf 1.}]
Find all $x \in {\mathbb Z}$ satisfying each of the following equations.
\begin{multicols}{2}
\begin{enumerate}

\item 
$3x \equiv 2 \pmod{ 7}$

\textbf{Answer:} The solutions are $x \equiv 3 \pmod{7}$.

\item
$5x + 1 \equiv 13 \pmod{ 23}$

\textbf{Answer:} The solutions are $x \equiv 4 \pmod{23}$.

\item
$5x + 1 \equiv 13 \pmod{ 26}$

\textbf{Answer:} The solutions are $x \equiv 4 \pmod{26}$.

\item
$9x \equiv 3 \pmod{ 5}$

\textbf{Answer:} The solutions are $x \equiv 2 \pmod{5}$.

\item
$5x \equiv 1 \pmod{ 6}$

\textbf{Answer:} The solutions are $x \equiv 5 \pmod{6}$.

\item
$3x \equiv 1 \pmod{ 6}$

\textbf{Answer:} The solutions are $x \equiv 3 \pmod{6}$.

\end{enumerate}
\end{multicols}

\medskip
%% 2 %%%%%%%%%%%%%%%%%%%%%%%%%%%%%%%%%%%%%%%%%%%%%%%%
\item[{\bf 2.}]
Which of the following multiplication tables defined on the set $G =
\{ a, b, c, d \}$ form a group? Support your answer in each case. 
\begin{multicols}{2}
\begin{enumerate}

\item
\[
\begin{array}{c|cccc}
\circ & a & b & c & d \\
\hline
a & a & c & d & a \\
b & b & b & c & d \\
c & c & d & a & b \\
d & d & a & b & c
\end{array}
\]

\textbf{Answer:} This table does not form a group because the operation is not associative.

\item
\[
\begin{array}{c|cccc}
\circ & a & b & c & d \\
\hline
a & a & b & c & d \\
b & b & a & d & c \\
c & c & d & a & b \\
d & d & c & b & a
\end{array}
\]
 
\textbf{Answer:} This table forms a group.

\item
\[
\begin{array}{c|cccc}
\circ & a & b & c & d \\
\hline
a & a & b & c & d \\
b & b & c & d & a \\
c & c & d & a & b \\
d & d & a & b & c
\end{array}
\]

\textbf{Answer:} This table forms a group.

\item
\[
\begin{array}{c|cccc}
\circ & a & b & c & d \\
\hline
a & a & b & c & d \\
b & b & a & c & d \\
c & c & b & a & d \\
d & d & d & b & c
\end{array}
\]

\textbf{Answer:} This table does not form a group because not every element has an inverse.

\end{enumerate}
\end{multicols}

\medskip

%% 3 %%%%%%%%%%%%%%%%%%%%%%%%%%%%%%%%%%%%%%%%%%%%%%%%
\item[{\bf 3.}]
Write out Cayley tables for groups formed by the symmetries of a
rectangle and for $({\mathbb Z}_4, +)$. How many elements are in each
group? Are the groups the same? Why or why not? 

\textbf{Answer:} The Cayley table for the symmetries of a rectangle is as follows:

\[
\begin{array}{c|cccc}
\circ & e & r & h & v \\
\hline
e & e & r & h & v \\
r & r & e & v & h \\
h & h & v & e & r \\
v & v & h & r & e
\end{array}
\]

The Cayley table for $({\mathbb Z}_4, +)$ is as follows:

\[
\begin{array}{c|cccc}
+ & 0 & 1 & 2 & 3 \\
\hline
0 & 0 & 1 & 2 & 3 \\
1 & 1 & 2 & 3 & 0 \\
2 & 2 & 3 & 0 & 1 \\
3 & 3 & 0 & 1 & 2
\end{array}
\]

Both groups have 4 elements, but they are not the same group. The group of symmetries of a rectangle is non-abelian, while $({\mathbb Z}_4, +)$ is an abelian group.

\medskip

%% 5 %%%%%%%%%%%%%%%%%%%%%%%%%%%%%%%%%%%%%%%%%%%%%%%%
\item[{\bf 5.}]
Describe the symmetries of a square and prove that the set of
symmetries is a group. Give a Cayley table for the symmetries. How
many ways can the vertices of a square be permuted?  Is each
permutation necessarily a symmetry of the square?  The symmetry group
of the square is denoted by $D_4$.

\textbf{Answer:} The symmetries of a square are the identity ($e$), rotations ($r$, $r^2$, $r^3$), and reflections ($h$, $v$, $d$, $a$). The Cayley table for $D_4$ is as follows:

\[
\begin{array}{c|cccccccc}
\circ & e & r & r^2 & r^3 & h & v & d & a \\
\hline
e & e & r & r^2 & r^3 & h & v & d & a \\
r & r & r^2 & r^3 & e & d & a & h & v \\
r^2 & r^2 & r^3 & e & r & v & d & a & h \\
r^3 & r^3 & e & r & r^2 & a & h & v & d \\
h & h & d & v & a & e & r^2 & r^3 & r \\
v & v & a & h & d & r^2 & e & r & r^3 \\
d & d & h & a & v & r^3 & r & e & r^2 \\
a & a & v & d & h & r & r^3 & r^2 & e
\end{array}
\]

There are 4! = 24 ways to permute the vertices of a square. Not every permutation is a symmetry of the square. Only the permutations that preserve the shape and orientation of the square are symmetries.

\medskip

%% 7 %%%%%%%%%%%%%%%%%%%%%%%%%%%%%%%%%%%%%%%%%%%%%%%%
\item[{\bf 7.}]
Let $S = {\mathbb R} \setminus \{ -1 \}$ and define a binary operation on
$S$ by $a \ast b = a + b +ab$. Prove that $(S, \ast)$ is an abelian
group.

\textbf{Answer:} To prove that $(S, \ast)$ is an abelian group, we need to show that it satisfies the group axioms: closure, associativity, identity, inverse, and commutativity.

\begin{itemize}
    \item \textbf{Closure:} For any $a, b \in S$, $a \ast b = a + b + ab$ is also in $S$ since $-1$ is excluded from $S$.
    \item \textbf{Associativity:} The operation $\ast$ is associative since $(a \ast b) \ast c = a + b + ab + c + (a + b + ab)c = a + b + c + ab + ac + bc + abc = a + (b + c + bc) + (a + b + ab)(b + c + bc) = a \ast (b \ast c)$ for all $a, b, c \in S$.
    \item \textbf{Identity:} The identity element is 0 since $a \ast 0 = a + 0 + a \cdot 0 = a$ for all $a \in S$.
    \item \textbf{Inverse:} For any $a \in S$, the inverse element is $-1 - a$ since $a \ast (-1 - a) = a + (-1 - a) + a(-1 - a) = -1$.
    \item \textbf{Commutativity:} The operation $\ast$ is commutative since $a \ast b = a + b + ab = b + a + ba = b \ast a$ for all $a, b \in S$.
\end{itemize}

Therefore, $(S, \ast)$ is an abelian group.

\medskip

%% 12 %%%%%%%%%%%%%%%%%%%%%%%%%%%%%%%%%%%%%%%%%%%%%%%%
\item[{\bf 12.}]
Let ${\mathbb Z}_2^n = \{ (a_1, a_2, \ldots, a_n) : a_i \in {\mathbb Z}_2
\}$. Define a binary operation on ${\mathbb Z}_2^n$ by
\[
(a_1, a_2, \ldots, a_n)
+
(b_1, b_2, \ldots, b_n)
=
(a_1+b_1, a_2+b_2, \ldots, a_n+b_n).
\]
Prove that ${\mathbb Z}_2^n$ is a group under this operation. This group
is important in algebraic coding theory. 

\textbf{Answer:} To prove that ${\mathbb Z}_2^n$ is a group under the given operation, we need