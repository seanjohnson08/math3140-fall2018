Certainly! Here's the updated file with answers added inline:

\documentclass[12pt,reqno]{amsart}
\usepackage[top=1.5cm, left=1.5cm,right=1.5cm,bottom=1.5cm]{geometry}
\renewcommand{\baselinestretch}{1.2}
\usepackage{amsmath}
\usepackage{amssymb}
\usepackage{color,hyperref,enumerate,multicol}
\definecolor{darkblue}{rgb}{0.0,0.0,0.3}
\hypersetup{colorlinks,breaklinks,
            linkcolor=darkblue,urlcolor=darkblue,
            anchorcolor=darkblue,citecolor=darkblue}
            
\usepackage{algorithm}
\usepackage{algorithmic}
\pagestyle{empty}
\newcommand{\N}{\ensuremath{\mathbb{N}}}
\newcommand{\Z}{\ensuremath{\mathbb{Z}}}
\newcommand{\R}{\ensuremath{\mathbb{R}}}
\newcommand{\meet}{\ensuremath{\wedge}}
\newcommand{\Meet}{\ensuremath{\bigwedge}}
\newcommand{\join}{\ensuremath{\vee}}
\renewcommand{\emptyset}{\ensuremath{\varnothing}}
\renewcommand{\subset}{\ensuremath{\subsetneq}}
\newcommand{\boldemph}{\emph}
\newcommand{\lcm}{\operatorname{lcm}}

\begin{document}
\thispagestyle{empty}

\noindent \textbf{Math 3140} \hfill {\bf Homework 3} 
\vskip1cm
\noindent {\bf Chapter 3:} 1bd, 2bd, 3, 5, 7, 12.  
\\
{\bf Due date:} Wednesday, 9/19

\medskip

\begin{enumerate}

%% 1 %%%%%%%%%%%%%%%%%%%%%%%%%%%%%%%%%%%%%%%%%%%%%%%%
\item[{\bf 1.}]
Find all $x \in {\mathbb Z}$ satisfying each of the following equations.
\begin{multicols}{2}
\begin{enumerate}

\item 
$3x \equiv 2 \pmod{ 7}$

\textbf{Answer:} $x \equiv 5 \pmod{7}$

\item
$5x + 1 \equiv 13 \pmod{ 23}$

\textbf{Answer:} $x \equiv 4 \pmod{23}$

\item
$5x + 1 \equiv 13 \pmod{ 26}$

\textbf{Answer:} $x \equiv 12 \pmod{26}$

\item
$9x \equiv 3 \pmod{ 5}$

\textbf{Answer:} $x \equiv 2 \pmod{5}$

\item
$5x \equiv 1 \pmod{ 6}$

\textbf{Answer:} $x \equiv 5 \pmod{6}$

\item
$3x \equiv 1 \pmod{ 6}$

\textbf{Answer:} $x \equiv 5 \pmod{6}$

\end{enumerate}
\end{multicols}

\medskip
%% 2 %%%%%%%%%%%%%%%%%%%%%%%%%%%%%%%%%%%%%%%%%%%%%%%%
\item[{\bf 2.}]
Which of the following multiplication tables defined on the set $G =
\{ a, b, c, d \}$ form a group? Support your answer in each case. 
\begin{multicols}{2}
\begin{enumerate}

\item
\[
\begin{array}{c|cccc}
\circ & a & b & c & d \\
\hline
a & a & c & d & a \\
b & b & b & c & d \\
c & c & d & a & b \\
d & d & a & b & c
\end{array}
\]

\textbf{Answer:} This table does not form a group because the operation is not associative. For example, $(a \circ b) \circ c = c \circ c = a$, but $a \circ (b \circ c) = a \circ d = b$.

\item
\[
\begin{array}{c|cccc}
\circ & a & b & c & d \\
\hline
a & a & b & c & d \\
b & b & a & d & c \\
c & c & d & a & b \\
d & d & c & b & a
\end{array}
\]
 
\textbf{Answer:} This table forms a group because it satisfies the group axioms. It is associative, every element has an inverse, and it has an identity element.

\item
\[
\begin{array}{c|cccc}
\circ & a & b & c & d \\
\hline
a & a & b & c & d \\
b & b & c & d & a \\
c & c & d & a & b \\
d & d & a & b & c
\end{array}
\]

\textbf{Answer:} This table does not form a group because the operation is not associative. For example, $(a \circ b) \circ c = c \circ c = a$, but $a \circ (b \circ c) = a \circ d = b$.

\item
\[
\begin{array}{c|cccc}
\circ & a & b & c & d \\
\hline
a & a & b & c & d \\
b & b & a & c & d \\
c & c & b & a & d \\
d & d & d & b & c
\end{array}
\]
 
\textbf{Answer:} This table does not form a group because not every element has an inverse. For example, there is no element $x$ such that $a \circ x = d$.

\end{enumerate}
\end{multicols}

\medskip

%% 3 %%%%%%%%%%%%%%%%%%%%%%%%%%%%%%%%%%%%%%%%%%%%%%%%
\item[{\bf 3.}]
Write out Cayley tables for groups formed by the symmetries of a
rectangle and for $({\mathbb Z}_4, +)$. How many elements are in each
group? Are the groups the same? Why or why not? 

\textbf{Answer:} The Cayley table for the symmetries of a rectangle is as follows:

\[
\begin{array}{c|cccc}
\circ & e & r & h & v \\
\hline
e & e & r & h & v \\
r & r & e & v & h \\
h & h & v & e & r \\
v & v & h & r & e
\end{array}
\]

The Cayley table for $({\mathbb Z}_4, +)$ is as follows:

\[
\begin{array}{c|cccc}
+ & 0 & 1 & 2 & 3 \\
\hline
0 & 0 & 1 & 2 & 3 \\
1 & 1 & 2 & 3 & 0 \\
2 & 2 & 3 & 0 & 1 \\
3 & 3 & 0 & 1 & 2
\end{array}
\]

Both groups have 4 elements. However, the groups are not the same because the operation in each group is different. The group of symmetries of a rectangle is formed by composition of symmetries, while $({\mathbb Z}_4, +)$ is formed by addition modulo 4.

\medskip

%% 5 %%%%%%%%%%%%%%%%%%%%%%%%%%%%%%%%%%%%%%%%%%%%%%%%
\item[{\bf 5.}]
Describe the symmetries of a square and prove that the set of
symmetries is a group. Give a Cayley table for the symmetries. How
many ways can the vertices of a square be permuted?  Is each
permutation necessarily a symmetry of the square?  The symmetry group
of the square is denoted by $D_4$.

\textbf{Answer:} The symmetries of a square are as follows:
\begin{itemize}
    \item The identity symmetry $e$ (no rotation or reflection).
    \item Rotations: $r$, $r^2$, $r^3$ (by 90, 180, and 270 degrees counterclockwise, respectively).
    \item Reflections: $h$ (horizontal reflection) and $v$ (vertical reflection).
    \item Diagonal reflections: $d_1$ (reflection across the main diagonal) and $d_2$ (reflection across the other diagonal).
\end{itemize}

To prove that the set of symmetries is a group, we need to show that it satisfies the group axioms. 

\begin{itemize}
    \item Closure: The composition of any two symmetries of a square is also a symmetry of a square.
    \item Associativity: The composition of symmetries is associative.
    \item Identity: The identity symmetry $e$ is the identity element.
    \item Inverses: Every symmetry has an inverse symmetry.
\end{itemize}

The Cayley table for the symmetries of a square is as follows:

\[
\begin{array}{c|cccccccc}
\circ & e & r & r^2 & r^3 & h & v & d_1 & d_2 \\
\hline
e & e & r & r^2 & r^3 & h & v & d_1 & d_2 \\
r & r & r^2 & r^3 & e & d_2 & d_1 & v & h \\
r^2 & r^2 & r^3 & e & r & v & h & d_2 & d_1 \\
r^3 & r^3 & e & r & r^2 & d_1 & d_2 & h & v \\
h & h & d_1 & v & d_2 & e & r^2 & r^3 & r \\
v & v & d_2 & h & d_1 & r^2 & e & r & r^3 \\
d_1 & d_1 & v & d_2 & h & r^3 & r & e & r^2 \\
d_2 & d_2 & h & d_1 & v & r & r^3 & r^2 & e \\
\end{array}
\]

There are $4! = 24$ ways to permute the vertices of a square. However, not every permutation is necessarily a symmetry of the square. Only the permutations that can be achieved by rotations and reflections are symmetries of the square.

\medskip

%% 7 %%%%%%%%%%%%%%%%%%%%%%%%%%%%%%%%%%%%%%%%%%%%%%%%
\item[{\bf 7.}]
Let $S = {\mathbb R} \setminus \{ -1 \}$ and define a binary operation on
$S$ by $a \ast b = a + b +ab$. Prove that $(S, \ast)$ is an abelian
group.

\textbf{Answer:} To prove that $(S, \ast)$ is an abelian group, we need to show that it satisfies the group axioms.

\begin{itemize}
    \item Closure: For any $a, b \in S$, $a \ast b = a + b + ab$ is also in $S$ because the sum and product of real numbers is always a real number.
    \item Associativity: The operation $\ast$ is associative because $(a \ast b) \ast c = a + b + ab + c + (a + b + ab)c = a + b + c + ab + ac + bc + abc = a + (b + c + bc) + (a + (b + c + bc))bc = a \ast (b \ast c)$.
    \item Identity: The identity element is 0 because $a \ast 0 = a + 0 + a \cdot 0 = a$ for any $a \in S$.
    \item Invers