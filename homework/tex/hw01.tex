% Example LaTeX document for GP111 - note % sign indicates a comment
\documentclass[12pt,reqno]{amsart}
\usepackage[top=1.5cm, left=1.5cm,right=1.5cm,bottom=1.5cm]{geometry}

\usepackage{amsmath}
\usepackage{amssymb}
\usepackage{color,hyperref,enumerate,multicol}
\definecolor{darkblue}{rgb}{0.0,0.0,0.3}
\hypersetup{colorlinks,breaklinks,
            linkcolor=darkblue,urlcolor=darkblue,
            anchorcolor=darkblue,citecolor=darkblue}
            
\usepackage{algorithm}
\usepackage{algorithmic}
\pagestyle{empty}
\newcommand{\N}{\ensuremath{\mathbb{N}}}
\newcommand{\Z}{\ensuremath{\mathbb{Z}}}
\newcommand{\R}{\ensuremath{\mathbb{R}}}
\newcommand{\meet}{\ensuremath{\wedge}}
\newcommand{\Meet}{\ensuremath{\bigwedge}}
\newcommand{\join}{\ensuremath{\vee}}
\renewcommand{\emptyset}{\ensuremath{\varnothing}}

\begin{document}
\thispagestyle{empty}

\noindent \textbf{Math 3140} \hskip4cm {\bf Homework 1} \hfill {\bf Fall 2018}
\vskip1cm
\noindent {\bf Chapter 1:}  1cd, 2bd, 3, 7, 20b, 22cd, 24bc(de), 25d, 28.  \\
{\bf Due date:} Wednesday, 9/5

\medskip

\begin{enumerate}[{\bf 1.}]

%% 1 %%%%%%%%%%%%%%%%%%%%%%%%%%%%%%%%%%%%%%%%%%%%%%%%
\item[{\bf 1.}]
Suppose that
\begin{align*}
A & = \{ x : x \in \mathbb N \text{ and } x \text{ is even} \}, \\
B & = \{x : x \in \mathbb N \text{ and } x \text{ is prime}\}, \\
C & = \{ x : x \in \mathbb N \text{ and } x \text{ is a multiple of 5}\}.
\end{align*}
Describe each of the following sets. 
\begin{multicols}{2}
\begin{enumerate}

\item
$A \cap B$ \\
\textbf{Answer:} $A \cap B = \emptyset$ (the empty set), since there are no numbers that are both even and prime.

\item
$B \cap C$ \\
\textbf{Answer:} $B \cap C = \{5\}$, since 5 is the only number that is both prime and a multiple of 5.

\item
$A \cup B$ \\
\textbf{Answer:} $A \cup B = \{2, 3, 5, 7, 11, 13, \ldots\}$, which is the set of all prime numbers.

\item
$A \cap (B \cup C)$ \\
\textbf{Answer:} $A \cap (B \cup C) = A$, since $B \cup C$ includes all prime numbers and multiples of 5, and $A$ only contains even numbers.

\end{enumerate}
\end{multicols}

\medskip  

%% 2 %%%%%%%%%%%%%%%%%%%%%%%%%%%%%%%%%%%%%%%%%%%%%%%%
\item[{\bf 2.}]
If $A = \{ a, b, c \}$, $B = \{ 1, 2, 3 \}$, $C = \{ x \}$, and 
$D = \emptyset$, list all of the elements in each of the following sets. 
\begin{multicols}{2}
\begin{enumerate}

\item
$A \times B$ \\
\textbf{Answer:} $A \times B = \{ (a, 1), (a, 2), (a, 3), (b, 1), (b, 2), (b, 3), (c, 1), (c, 2), (c, 3) \}$

\item
$B \times A$ \\
\textbf{Answer:} $B \times A = \{ (1, a), (1, b), (1, c), (2, a), (2, b), (2, c), (3, a), (3, b), (3, c) \}$

\item
$A \times B \times C$ \\
\textbf{Answer:} $A \times B \times C = \{ (a, 1, x), (a, 2, x), (a, 3, x), (b, 1, x), (b, 2, x), (b, 3, x), (c, 1, x), (c, 2, x), (c, 3, x) \}$

\item
$A \times D$ \\
\textbf{Answer:} $A \times D = \emptyset$, since $D$ is the empty set.

\end{enumerate}
\end{multicols}

\medskip  

%% 3 %%%%%%%%%%%%%%%%%%%%%%%%%%%%%%%%%%%%%%%%%%%%%%%%
\item[{\bf 3.}]
Find an example of two nonempty sets $A$ and $B$ for which $A \times B = B \times A$. \\
\textbf{Answer:} Let $A = \{1\}$ and $B = \{2\}$. Then $A \times B = \{(1, 2)\}$ and $B \times A = \{(2, 1)\}$, so $A \times B = B \times A$.

\medskip
 
%% 7 %%%%%%%%%%%%%%%%%%%%%%%%%%%%%%%%%%%%%%%%%%%%%%%%
\item[{\bf 7.}]
Prove $A \cap (B \cup C) = (A \cap B) \cup (A \cap C)$. \\
\textbf{Answer:} To prove this equality, we need to show that every element in $A \cap (B \cup C)$ is also in $(A \cap B) \cup (A \cap C)$, and vice versa.

Let $x$ be an arbitrary element in $A \cap (B \cup C)$. This means that $x$ is in both $A$ and $B \cup C$. Since $x$ is in $B \cup C$, it must be either in $B$ or in $C$. 

Case 1: $x$ is in $B$. In this case, $x$ is in both $A$ and $B$, so $x$ is in $A \cap B$. Therefore, $x$ is also in $(A \cap B) \cup (A \cap C)$.

Case 2: $x$ is in $C$. In this case, $x$ is in both $A$ and $C$, so $x$ is in $A \cap C$. Therefore, $x$ is also in $(A \cap B) \cup (A \cap C)$.

Since $x$ was an arbitrary element in $A \cap (B \cup C)$, we have shown that every element in $A \cap (B \cup C)$ is also in $(A \cap B) \cup (A \cap C)$.

Now, let $y$ be an arbitrary element in $(A \cap B) \cup (A \cap C)$. This means that $y$ is either in $A \cap B$ or in $A \cap C$.

Case 1: $y$ is in $A \cap B$. In this case, $y$ is in both $A$ and $B$, so $y$ is in $A$. Therefore, $y$ is also in $A \cap (B \cup C)$.

Case 2: $y$ is in $A \cap C$. In this case, $y$ is in both $A$ and $C$, so $y$ is in $A$. Therefore, $y$ is also in $A \cap (B \cup C)$.

Since $y$ was an arbitrary element in $(A \cap B) \cup (A \cap C)$, we have shown that every element in $(A \cap B) \cup (A \cap C)$ is also in $A \cap (B \cup C)$.

Therefore, $A \cap (B \cup C) = (A \cap B) \cup (A \cap C)$.

\medskip

% 20 %%%%%%%%%%%%%%%%%%%%%%%%%%%%%%%%%%%%%%%%%%%%%%%%%
\item[{\bf 20.}]
\begin{enumerate}
  
\item
Define a function $f: {\mathbb N} \rightarrow {\mathbb N}$ that is one-to-one but not onto. \\
\textbf{Answer:} One example of such a function is $f(x) = 2x$. This function is one-to-one because if $f(x) = f(y)$, then $2x = 2y$, which implies $x = y$. However, this function is not onto because there are natural numbers that are not in the range of $f$, such as 1.

\item
Define a function $f: {\mathbb N} \rightarrow {\mathbb N}$ that is onto but not one-to-one. \\
\textbf{Answer:} One example of such a function is $f(x) = \lceil x/2 \rceil$. This function is onto because for every natural number $y$, we can find a natural number $x$ such that $f(x) = y$. However, this function is not one-to-one because, for example, $f(1) = f(2) = 1$.

\end{enumerate}

\medskip
 
% 22 %%%%%%%%%%%%%%%%%%%%%%%%%%%%%%%%%%%%%%%%%%%%%%%%%
\item[{\bf 22.}]
Let $f : A \rightarrow B$ and $g : B \rightarrow C$ be maps.
\begin{enumerate}
 
\item
If $f$ and $g$ are both one-to-one functions, show that $g \circ f$ is one-to-one. \\
\textbf{Answer:} To show that $g \circ f$ is one-to-one, we need to prove that if $(g \circ f)(x) = (g \circ f)(y)$, then $x = y$.

Let $(