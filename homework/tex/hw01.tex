Certainly! Here's the updated file with answers added inline:

```tex
% Example LaTeX document for GP111 - note % sign indicates a comment
\documentclass[12pt,reqno]{amsart}
\usepackage[top=1.5cm, left=1.5cm,right=1.5cm,bottom=1.5cm]{geometry}

\usepackage{amsmath}
\usepackage{amssymb}
\usepackage{color,hyperref,enumerate,multicol}
\definecolor{darkblue}{rgb}{0.0,0.0,0.3}
\hypersetup{colorlinks,breaklinks,
            linkcolor=darkblue,urlcolor=darkblue,
            anchorcolor=darkblue,citecolor=darkblue}
            
\usepackage{algorithm}
\usepackage{algorithmic}
\pagestyle{empty}
\newcommand{\N}{\ensuremath{\mathbb{N}}}
\newcommand{\Z}{\ensuremath{\mathbb{Z}}}
\newcommand{\R}{\ensuremath{\mathbb{R}}}
\newcommand{\meet}{\ensuremath{\wedge}}
\newcommand{\Meet}{\ensuremath{\bigwedge}}
\newcommand{\join}{\ensuremath{\vee}}
\renewcommand{\emptyset}{\ensuremath{\varnothing}}

\begin{document}
\thispagestyle{empty}

\noindent \textbf{Math 3140} \hskip4cm {\bf Homework 1} \hfill {\bf Fall 2018}
\vskip1cm
\noindent {\bf Chapter 1:}  1cd, 2bd, 3, 7, 20b, 22cd, 24bc(de), 25d, 28.  \\
{\bf Due date:} Wednesday, 9/5

\medskip

\begin{enumerate}[{\bf 1.}]

%% 1 %%%%%%%%%%%%%%%%%%%%%%%%%%%%%%%%%%%%%%%%%%%%%%%%
\item[{\bf 1.}]
Suppose that
\begin{align*}
A & = \{ x : x \in \mathbb N \text{ and } x \text{ is even} \}, \\
B & = \{x : x \in \mathbb N \text{ and } x \text{ is prime}\}, \\
C & = \{ x : x \in \mathbb N \text{ and } x \text{ is a multiple of 5}\}.
\end{align*}
Describe each of the following sets. 
\begin{multicols}{2}
\begin{enumerate}

\item
$A \cap B$ \textbf{Answer:} The set of natural numbers that are both even and prime.

\item
$B \cap C$ \textbf{Answer:} The set of natural numbers that are both prime and multiples of 5.

\item
$A \cup B$ \textbf{Answer:} The set of natural numbers that are either even or prime.

\item
$A \cap (B \cup C)$ \textbf{Answer:} The set of natural numbers that are even and either prime or multiples of 5.

\end{enumerate}
\end{multicols}

\medskip  

%% 2 %%%%%%%%%%%%%%%%%%%%%%%%%%%%%%%%%%%%%%%%%%%%%%%%
\item[{\bf 2.}]
If $A = \{ a, b, c \}$, $B = \{ 1, 2, 3 \}$, $C = \{ x \}$, and 
$D = \emptyset$, list all of the elements in each of the following sets. 
\begin{multicols}{2}
\begin{enumerate}

\item
$A \times B$ \textbf{Answer:} $\{(a,1), (a,2), (a,3), (b,1), (b,2), (b,3), (c,1), (c,2), (c,3)\}$

\item
$B \times A$ \textbf{Answer:} $\{(1,a), (1,b), (1,c), (2,a), (2,b), (2,c), (3,a), (3,b), (3,c)\}$

\item
$A \times B \times C$ \textbf{Answer:} $\{(a,1,x), (a,2,x), (a,3,x), (b,1,x), (b,2,x), (b,3,x), (c,1,x), (c,2,x), (c,3,x)\}$

\item
$A \times D$ \textbf{Answer:} $\emptyset$

\end{enumerate}
\end{multicols}

\medskip  

%% 3 %%%%%%%%%%%%%%%%%%%%%%%%%%%%%%%%%%%%%%%%%%%%%%%%
\item[{\bf 3.}]
Find an example of two nonempty sets $A$ and $B$ for which $A \times B = B \times A$. \textbf{Answer:} Let $A = \{1\}$ and $B = \{2\}$.

\medskip

 
%% 7 %%%%%%%%%%%%%%%%%%%%%%%%%%%%%%%%%%%%%%%%%%%%%%%%
\item[{\bf 7.}]
Prove $A \cap (B \cup C) = (A \cap B) \cup (A \cap C)$. \textbf{Answer:} To prove this, we need to show that every element in $A \cap (B \cup C)$ is also in $(A \cap B) \cup (A \cap C)$, and vice versa.

Let $x$ be an arbitrary element in $A \cap (B \cup C)$. This means that $x$ is in both $A$ and $B \cup C$. Since $x$ is in $B \cup C$, it must be either in $B$ or in $C$. In either case, $x$ is in $A \cap B$ or $x$ is in $A \cap C$. Therefore, $x$ is in $(A \cap B) \cup (A \cap C)$.

Now, let $y$ be an arbitrary element in $(A \cap B) \cup (A \cap C)$. This means that $y$ is either in $A \cap B$ or in $A \cap C$. In either case, $y$ is in $A$. Since $y$ is in $A$ and $y$ is either in $B$ or in $C$, $y$ is in $A \cap (B \cup C)$. Therefore, $y$ is in $A \cap (B \cup C)$.

Since every element in $A \cap (B \cup C)$ is also in $(A \cap B) \cup (A \cap C)$, and vice versa, we can conclude that $A \cap (B \cup C) = (A \cap B) \cup (A \cap C)$.

\medskip

% 20 %%%%%%%%%%%%%%%%%%%%%%%%%%%%%%%%%%%%%%%%%%%%%%%%%
\item[{\bf 20.}]
\begin{enumerate}
  
\item
Define a function $f: {\mathbb N} \rightarrow {\mathbb N}$ that is one-to-one but not onto. \textbf{Answer:} $f(x) = 2x$

\item
Define a function $f: {\mathbb N} \rightarrow {\mathbb N}$ that is onto but not one-to-one. \textbf{Answer:} $f(x) = \lceil \frac{x}{2} \rceil$

\end{enumerate}

\medskip
 
% 22 %%%%%%%%%%%%%%%%%%%%%%%%%%%%%%%%%%%%%%%%%%%%%%%%%
\item[{\bf 22.}]
Let $f : A \rightarrow B$ and $g : B \rightarrow C$ be maps.
\begin{enumerate}
 
\item
If $f$ and $g$ are both one-to-one functions, show that $g \circ f$
is one-to-one. \textbf{Answer:} To prove that $g \circ f$ is one-to-one, we need to show that for any $x_1, x_2 \in A$, if $g(f(x_1)) = g(f(x_2))$, then $x_1 = x_2$.

Since $f$ is one-to-one, $f(x_1) = f(x_2)$ implies $x_1 = x_2$. Similarly, since $g$ is one-to-one, $g(f(x_1)) = g(f(x_2))$ implies $f(x_1) = f(x_2)$. Therefore, $g(f(x_1)) = g(f(x_2))$ implies $x_1 = x_2$, which means that $g \circ f$ is one-to-one.

\item
If $g \circ f$ is onto, show that $g$ is onto. \textbf{Answer:} To prove that $g$ is onto, we need to show that for any $y \in C$, there exists an $x \in A$ such that $g(x) = y$.

Since $g \circ f$ is onto, for any $y \in C$, there exists an $x \in A$ such that $(g \circ f)(x) = y$. Since $(g \circ f)(x) = g(f(x))$, we have $g(f(x)) = y$. Therefore, for any $y \in C$, there exists an $x \in A$ such that $g(x) = y$, which means that $g$ is onto.

\item
If $g \circ f$ is one-to-one, show that $f$ is one-to-one. \textbf{Answer:} To prove that $f$ is one-to-one, we need to show that for any $x_1, x_2 \in A$, if $f(x_1) = f(x_2)$, then $x_1 = x_2$.

Since $g \circ f$ is one-to-one, $g(f(x_1)) = g(f(x_2))$ implies $x_1 = x_2$. Therefore, if $f(x_1) = f(x_2)$, then $x_1 = x_2$, which means that $f$ is one-to-one.

\item
If $g \circ f$ is one-to-one and $f$ is onto, show that $g$ is one-to-one. \textbf{Answer:} To prove that $g$ is one-to-one, we need to show that for any $y_1, y_2 \in B$, if $g(y_1) = g(y_2)$, then $y_1 = y_2$