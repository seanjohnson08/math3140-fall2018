Certainly! Here are the answers to the homework questions:

1. To show that the two definitions of a homomorphism are equivalent, we need to prove that condition (1) implies conditions (2) and (3). 

Assume that $\varphi: G \rightarrow H$ satisfies $\varphi(x \cdot y) = \varphi(x) \circ \varphi(y)$ for all $x, y \in G$. 

To prove condition (3), let $e_G$ and $e_H$ be the identity elements of $G$ and $H$, respectively. We have:
$$\varphi(e_G) = \varphi(e_G \cdot e_G) = \varphi(e_G) \circ \varphi(e_G).$$
Multiplying both sides by $\varphi(e_G)^{-1}$ on the right, we get $\varphi(e_G) = e_H$.

To prove condition (2), let $x \in G$. We have:
$$\varphi(\invG(x)) = \varphi(x \cdot x^{-1}) = \varphi(x) \circ \varphi(x^{-1}).$$
Multiplying both sides by $\varphi(x)^{-1}$ on the right, we get $\varphi(\invG(x)) = \invH(\varphi(x))$.

Therefore, condition (1) implies conditions (2) and (3), and the two definitions of a homomorphism are equivalent.

2. The definition of an isomorphism as given in the file is not appropriate for posets. Consider the posets $\bP$ and $\bQ$ shown in the file. Let $\varphi: \bP \rightarrow \bQ$ be a function that maps $a$ to $0$, $b$ to $1$, and $c$ to $2$. This function is an order-preserving map, but it does not have an inverse that is also an order-preserving map. Therefore, $\varphi$ does not satisfy the definition of an isomorphism given in the file.

3. The $n$th roots of unity are isomorphic to ${\mathbb Z}_n$. To prove this, we can define a function $\varphi: {\mathbb Z}_n \rightarrow \