\documentclass[12pt,reqno]{amsart}
\usepackage[top=2cm, left=2cm,right=2cm,bottom=2cm]{geometry}
\renewcommand{\baselinestretch}{1.2}
\usepackage{amsmath}
\usepackage{xspace}
\usepackage{amssymb}
\usepackage{color,hyperref,enumerate,multicol}
\definecolor{darkblue}{rgb}{0.0,0.0,0.3}
\hypersetup{colorlinks,breaklinks,
            linkcolor=darkblue,urlcolor=darkblue,
            anchorcolor=darkblue,citecolor=darkblue}
            
\usepackage{algorithm}
\usepackage{algorithmic}
\pagestyle{empty}
\newcommand{\N}{\ensuremath{\mathbb{N}}}
\newcommand{\Z}{\ensuremath{\mathbb{Z}}}
\newcommand{\R}{\ensuremath{\mathbb{R}}}
\newcommand{\meet}{\ensuremath{\wedge}}
\newcommand{\Meet}{\ensuremath{\bigwedge}}
\newcommand{\join}{\ensuremath{\vee}}
\renewcommand{\emptyset}{\ensuremath{\varnothing}}
\renewcommand{\subset}{\ensuremath{\subsetneq}}
\newcommand{\boldemph}{\emph}
\newcommand{\lcm}{\operatorname{lcm}}
\newcommand{\subject}{MATH\xspace}
\newcommand{\coursenumber}{3140\xspace}
\newcommand{\semester}{Fall 2018\xspace}
\newcommand{\probskip}{\vskip1cm}

\begin{document}
\thispagestyle{empty}

\noindent \textbf{\subject \coursenumber Homework 8}

\medskip

\noindent {\bf Exercises:} 1 below and Judson: 6.5bd, 6.11ade, 6.14, 6.16, 6.18\\
{\bf Due date:} Friday, 10/26

\bigskip

\begin{enumerate}[{\bf 1.}]

%% 1 %%%%%%%%%%%%%%%%%%%%%%%%%%%%%%%%%%%%%%%%%%%%%%%%
\item
Prove or disprove the following: 
\begin{enumerate}
\item 
There exists a group $G$ of order $|G| = 8$ with an element
$g \in G$ of order $|g|=3$.

\textbf{Answer:} Yes, such a group exists. One example is the group of symmetries of an equilateral triangle.

\item
If $H$ and $K$ are subgroups of a group $G$ with $|H|=2$ and $|K|=3$, 
  then $|G|\geq 6$.

\textbf{Answer:} No, this statement is not true. For example, consider the group $G = \mathbb{Z}_4$ with $H = \{0, 2\}$ and $K = \{0, 3\}$. Here, $|H| = 2$ and $|K| = 2$, but $|G| = 4$.

\item
Every subgroup of the integers has finite index.

\textbf{Answer:} Yes, this statement is true. Every subgroup of the integers has finite index, which means that the quotient group is finite.

\item 
Every subgroup of the integers has finite order.

\textbf{Answer:} No, this statement is not true. The subgroup of even integers, for example, does not have finite order.
\end{enumerate}
\probskip

%% 5 %%%%%%%%%%%%%%%%%%%%%%%%%%%%%%%%%%%%%%%%%%%%%%%%
\item[{\bf 6.5.}]
In each case below, list the left cosets of $H$ in $G$.
\begin{enumerate}
\item[{\bf b.}]
$G = U(8)$, $H = \langle 3 \rangle$.

\textbf{Answer:} The left cosets of $H$ in $G$ are:
\[
H = \langle 3 \rangle = \{1, 3, 5, 7\}
\]
\[
2H = \{2, 6\}
\]
\[
4H = \{4, 0\}
\]
\[
6H = \{6, 2\}
\]
\item[{\bf c.}]
$G = S_4$, $H = A_4$.

\textbf{Answer:} The left cosets of $H$ in $G$ are:
\[
H = A_4
\]
\[
(12)H = \{(12), (34)\}
\]
\[
(13)H = \{(13), (24)\}
\]
\[
(14)H = \{(14), (23)\}
\]
\[
(123)H = \{(123), (243)\}
\]
\[
(132)H = \{(132), (234)\}
\]
\[
(124)H = \{(124), (142)\}
\]
\[
(142)H = \{(142), (124)\}
\]
\[
(134)H = \{(134), (143)\}
\]
\[
(143)H = \{(143), (134)\}
\]
\[
(234)H = \{(234), (132)\}
\]
\[
(243)H = \{(243), (123)\}
\]
\[
(12)(34)H = \{(12)(34), (34)(12)\}
\]
\[
(13)(24)H = \{(13)(24), (24)(13)\}
\]
\[
(14)(23)H = \{(14)(23), (23)(14)\}
\]
\[
(1234)H = \{(1234), (1432)\}
\]
\[
(1243)H = \{(1243), (1423)\}
\]
\[
(1324)H = \{(1324), (2341)\}
\]
\[
(1342)H = \{(1342), (2431)\}
\]
\[
(1423)H = \{(1423), (1243)\}
\]
\[
(1432)H = \{(1432), (1234)\}
\]
\[
(2341)H = \{(2341), (1324)\}
\]
\[
(2431)H = \{(2431), (1342)\}
\]
\[
(123)(45)H = \{(123)(45), (243)(15)\}
\]
\[
(132)(45)H = \{(132)(45), (234)(15)\}
\]
\[
(124)(35)H = \{(124)(35), (142)(15)\}
\]
\[
(142)(35)H = \{(142)(35), (124)(15)\}
\]
\[
(134)(25)H = \{(134)(25), (143)(15)\}
\]
\[
(143)(25)H = \{(143)(25), (134)(15)\}
\]
\[
(234)(15)H = \{(234)(15), (132)(45)\}
\]
\[
(243)(15)H = \{(243)(15), (123)(45)\}
\]
\[
(1234)(5)H = \{(1234)(5), (1432)(5)\}
\]
\[
(1243)(5)H = \{(1243)(5), (1423)(5)\}
\]
\[
(1324)(5)H = \{(1324)(5), (2341)(5)\}
\]
\[
(1342)(5)H = \{(1342)(5), (2431)(5)\}
\]
\[
(1423)(5)H = \{(1423)(5), (1243)(5)\}
\]
\[
(1432)(5)H = \{(1432)(5), (1234)(5)\}
\]
\[
(2341)(5)H = \{(2341)(5), (1324)(5)\}
\]
\[
(2431)(5)H = \{(2431)(5), (1342)(5)\}
\]
\end{enumerate}

\probskip

%% 11 %%%%%%%%%%%%%%%%%%%%%%%%%%%%%%%%%%%%%%%%%%%%%%%%
\item[{\bf 6.11.}] 
Let $H$ be a subgroup of a group $G$ and suppose that $g_1, g_2 \in G$.  Prove
that the following conditions are equivalent: 
\begin{enumerate}
 
\item[(a)]
$g_1 H = g_2 H$
 
\item[(d)]
$g_2 \in g_1 H$
 
\item[(e)]
$g_1^{-1} g_2 \in H$
\end{enumerate}

\textbf{Answer:} We will prove the equivalence of the conditions (a), (d), and (e) by showing that each condition implies the other two.

First, assume that $g_1 H = g_2 H$. This means that for any $h \in H$, there exists $h' \in H$ such that $g_1 h = g_2 h'$. Multiplying both sides by $g_1^{-1}$ on the left, we get $h = g_1^{-1} g_2 h'$. Since $h$ and $h'$ are arbitrary elements of $H$, this shows that $g_1^{-1} g_2 \in H$. Therefore, condition (a) implies condition (e).

Next, assume that $g_2 \in g_1 H$. This means that there exists $h \in H$ such that $g_2 = g_1 h$. Multiplying both sides by $g_1^{-1}$ on the right, we get $g_1^{-1} g_2 = h$. Since $h$ is an element of $H$, this shows that $g_1^{-1} g_2 \in H$. Therefore, condition (d) implies condition (e).

Finally, assume that $g_1^{-1} g_2 \in H$. This means that there exists $h \in H$ such that $g_1^{-1} g_2 = h$. Multiplying both sides by $g_1$ on the left, we get $g_2 = g_1 h$. This shows that $g_2$ is an element of the set $g_1 H$. Therefore, condition (e) implies condition (d).

Since we have shown that each condition implies the other two, the conditions (a), (d), and (e) are equivalent.

\probskip

\item[{\bf 6.14}] Let $G$ be a group and suppose $g\in G$, $n> 0$, and $g^n = e$. 
Show that the order of $g$ divides $n$.

\textbf{Answer:} Let $m$ be the order of $g$. By definition, $m$ is the smallest positive integer such that $g^m = e$. Since $g^n = e$, we can write $n = mq + r$, where $q$ is a non-negative integer and $0 \leq r < m$. Then, we have:
\[
g^n = g^{mq + r} = (g^m)^q g^r = e^q g^r = g^r
\]
Since $g^n = g^r$, and $m$ is the smallest positive integer such that $g^m = e$, it follows that $r = 0$. Therefore, $n = mq$, which means that $m$ divides $n$.

\probskip


%% 16 %%%%%%%%%%%%%%%%%%%%%%%%%%%%%%%%%%%%%%%%%%%%%%%%
\item[{\bf 6.16.}] 
If $|G| = 2n$, prove that the number of elements of order 2 is odd.  Use this
result to show that $G$ must contain a subgroup of order 2. 

\textbf{Answer:} Let $S$ be the set of elements in $G$ that have order 2. We want to show that $|S|$ is odd.

Consider the set $G \setminus S$, which consists of all elements in $G$ that do not have order 2. Each element in $G \setminus S$ has order 1 or order greater than 2. Since the identity element $e$ has order 1, there are $2n - 1$ elements in $G \setminus S$ that have order greater than 2.

Now, let's count the number of pairs $(g, g^{-1})$ where $g \in G \setminus S$. Since $g$ and $g^{-1}$ are distinct elements, each pair contributes 2 elements to the count. Therefore, the total number of elements in $G \setminus S$ is $2 \cdot (2n - 1) = 4n - 2$.

Since $|G| = 2n$, the number of elements in $S$ is $|G| - |G \setminus S| = 2n - (4n - 2) = 2 - 2n$. This means that $|S|$ is odd.

Now, since $|S|$ is odd, there must exist an element $g \in S$ such that $g \neq e$. This element $g$ has order 2, so $\langle g \rangle$ is a subgroup of $G$ of order 2.

\probskip

%% 18 %%%%%%%%%%%%%%%%%%%%%%%%%%%%%%%%%%%%%%%%%%%%%%%%
\item[{\bf 6.18.}] 
If $[G : H]