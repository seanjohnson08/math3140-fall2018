Sure! Here's the updated file with answers added:

\documentclass[12pt,reqno]{amsart}
\usepackage[top=2cm, left=2cm,right=2cm,bottom=2cm]{geometry}
\renewcommand{\baselinestretch}{1.2}
\usepackage{amsmath}
\usepackage{xspace}
\usepackage{amssymb}
\usepackage{color,hyperref,enumerate,multicol}
\definecolor{darkblue}{rgb}{0.0,0.0,0.3}
\hypersetup{colorlinks,breaklinks,
            linkcolor=darkblue,urlcolor=darkblue,
            anchorcolor=darkblue,citecolor=darkblue}
            
\usepackage{algorithm}
\usepackage{algorithmic}
\pagestyle{empty}
\newcommand{\N}{\ensuremath{\mathbb{N}}}
\newcommand{\Z}{\ensuremath{\mathbb{Z}}}
\newcommand{\R}{\ensuremath{\mathbb{R}}}
\newcommand{\meet}{\ensuremath{\wedge}}
\newcommand{\Meet}{\ensuremath{\bigwedge}}
\newcommand{\join}{\ensuremath{\vee}}
\renewcommand{\emptyset}{\ensuremath{\varnothing}}
\renewcommand{\subset}{\ensuremath{\subsetneq}}
\newcommand{\boldemph}{\emph}
\newcommand{\lcm}{\operatorname{lcm}}
\newcommand{\subject}{MATH\xspace}
\newcommand{\coursenumber}{3140\xspace}
\newcommand{\semester}{Fall 2018\xspace}
\newcommand{\probskip}{\vskip1cm}

\begin{document}
\thispagestyle{empty}

\noindent \textbf{\subject \coursenumber Homework 8}

\medskip

\noindent {\bf Exercises:} 1 below and Judson: 6.5bd, 6.11ade, 6.14, 6.16, 6.18\\
{\bf Due date:} Friday, 10/26

\bigskip

\begin{enumerate}[{\bf 1.}]

%% 1 %%%%%%%%%%%%%%%%%%%%%%%%%%%%%%%%%%%%%%%%%%%%%%%%
\item
Prove or disprove the following: 
\begin{enumerate}
\item 
There exists a group $G$ of order $|G| = 8$ with an element
$g \in G$ of order $|g|=3$.

\textbf{Answer:} This statement is true. One example of such a group is the cyclic group $G = \langle g \rangle$ where $g$ is an element of order 3. In this case, $|G| = 8$ and $|g| = 3$.

\item
If $H$ and $K$ are subgroups of a group $G$ with $|H|=2$ and $|K|=3$, 
  then $|G|\geq 6$.

\textbf{Answer:} This statement is true. By Lagrange's Theorem, the order of a subgroup must divide the order of the group. Since $|H| = 2$ and $|K| = 3$, the orders of the subgroups divide the order of $G$. Therefore, $|G|$ must be at least the least common multiple of 2 and 3, which is 6.

\item
Every subgroup of the integers has finite index.

\textbf{Answer:} This statement is true. The index of a subgroup $H$ in a group $G$ is defined as the number of distinct left cosets of $H$ in $G$. In the case of the subgroup of integers, the index is infinite since there are infinitely many distinct left cosets (e.g., $H$, $H+1$, $H+2$, etc.). Therefore, the statement is false.

\item 
Every subgroup of the integers has finite order.

\textbf{Answer:} This statement is true. A subgroup of the integers is of the form $n\mathbb{Z}$ for some integer $n$. The order of this subgroup is equal to $|n|$, which is finite. Therefore, every subgroup of the integers has finite order.

\end{enumerate}
\probskip

%% 5 %%%%%%%%%%%%%%%%%%%%%%%%%%%%%%%%%%%%%%%%%%%%%%%%
\item[{\bf 6.5.}]
In each case below, list the left cosets of $H$ in $G$.
\begin{enumerate}
\item[{\bf b.}]
$G = U(8)$, $H = \langle 3 \rangle$.

\textbf{Answer:} The left cosets of $H$ in $G$ are:
\[
H = \langle 3 \rangle = \{1, 3, 5, 7\}
\]
\[
2H = \{2, 6\}
\]
\[
4H = \{4, 0\}
\]
\[
6H = \{6, 2\}
\]
\[
0H = \{0, 4\}
\]
\[
\text{Note: } 1H = H \text{ since } 1 \in H.
\]

\item[{\bf c.}]
$G = S_4$, $H = A_4$.

\textbf{Answer:} The left cosets of $H$ in $G$ are:
\[
H = A_4
\]
\[
(12)H = \{(12), (34)\}
\]
\[
(13)H = \{(13), (24)\}
\]
\[
(14)H = \{(14), (23)\}
\]
\[
(23)H = \{(23), (14)\}
\]
\[
(24)H = \{(24), (13)\}
\]
\[
(34)H = \{(34), (12)\}
\]
\[
(123)H = \{(123), (234)\}
\]
\[
(132)H = \{(132), (243)\}
\]
\[
(124)H = \{(124), (143)\}
\]
\[
(142)H = \{(142), (413)\}
\]
\[
(134)H = \{(134), (142)\}
\]
\[
(143)H = \{(143), (124)\}
\]
\[
(234)H = \{(234), (123)\}
\]
\[
(243)H = \{(243), (132)\}
\]
\[
(12)(34)H = \{(12)(34), (13)(24)\}
\]
\[
(13)(24)H = \{(13)(24), (12)(34)\}
\]
\[
(14)(23)H = \{(14)(23), (23)(14)\}
\]
\[
(23)(14)H = \{(23)(14), (14)(23)\}
\]
\[
(24)(13)H = \{(24)(13), (34)(12)\}
\]
\[
(34)(12)H = \{(34)(12), (24)(13)\}
\]
\[
(1234)H = \{(1234), (2431)\}
\]
\[
(1243)H = \{(1243), (1423)\}
\]
\[
(1324)H = \{(1324), (4132)\}
\]
\[
(1342)H = \{(1342), (1423)\}
\]
\[
(1423)H = \{(1423), (1243)\}
\]
\[
(1432)H = \{(1432), (2431)\}
\]
\[
(2134)H = \{(2134), (2431)\}
\]
\[
(2143)H = \{(2143), (1423)\}
\]
\[
(2314)H = \{(2314), (4132)\}
\]
\[
(2341)H = \{(2341), (2431)\}
\]
\[
(2413)H = \{(2413), (1423)\}
\]
\[
(2431)H = \{(2431), (1243)\}
\]
\[
(3124)H = \{(3124), (2431)\}
\]
\[
(3142)H = \{(3142), (1423)\}
\]
\[
(3214)H = \{(3214), (4132)\}
\]
\[
(3241)H = \{(3241), (2431)\}
\]
\[
(3412)H = \{(3412), (1423)\}
\]
\[
(3421)H = \{(3421), (1243)\}
\]
\[
(4123)H = \{(4123), (2431)\}
\]
\[
(4132)H = \{(4132), (1423)\}
\]
\[
(4213)H = \{(4213), (4132)\}
\]
\[
(4231)H = \{(4231), (2431)\}
\]
\[
(4312)H = \{(4312), (1423)\}
\]
\[
(4321)H = \{(4321), (1243)\}
\]
\[
\text{Note: } eH = H \text{ since } e \in H.
\]

\end{enumerate}

\probskip

%% 11 %%%%%%%%%%%%%%%%%%%%%%%%%%%%%%%%%%%%%%%%%%%%%%%%
\item[{\bf 6.11.}] 
Let $H$ be a subgroup of a group $G$ and suppose that $g_1, g_2 \in G$.  Prove
that the following conditions are equivalent: 
\begin{enumerate}
 
\item[(a)]
$g_1 H = g_2 H$
 
\item[(d)]
$g_2 \in g_1 H$
 
\item[(e)]
$g_1^{-1} g_2 \in H$
\end{enumerate}

\textbf{Answer:} To prove the equivalence of the conditions, we need to show that (a) implies (d), (d) implies (e), and (e) implies (a).

\textbf{(a) implies (d):} Suppose $g_1 H = g_2 H$. We want to show that $g_2 \in g_1 H$. Since $g_1 H = g_2 H$, this means that for any $h \in H$, there exists $h' \in H$ such that $g_1 h = g_2 h'$. In particular, if we choose $h = e$, the identity element of $H$, we have $g_1 e = g_2 h'$, which implies $g_1 = g_2 h'$. Therefore, $g_2 \in g_1 H$.

\textbf{(d) implies (e):} Suppose $g_2 \in g_1 H$. We want to show that $g_1^{-1} g_2 \in H$. Since $g_2 \in g_1 H$, there exists $h \in H$ such that $g_2 = g_1 h$. Multiplying both sides by $g_1^{-1}$ on the left, we have $g_1^{-1} g_2 = h$. Since $h \in H$, this implies $g_1^{-1} g_2 \in H$.

\textbf{(e) implies (a):} Suppose $g_1^{-1} g_2 \in H$. We want to show that $g_1 H = g_2 H$. To prove this, we need to show that every element in $g_1 H$ is also in $g_2 H$, and vice versa.

First, let $x \in g_1 H$. This means that there exists $h \in H$ such that $x = g_1 h$. We want to show that $x \in g_2 H$. Since $g_1^{-1} g_2 \in H$, we can rewrite $x$ as $x = g_1 (g_1^{-1} g_2) h$. By associativity, this becomes $x = (g_1 g_1^{-1}) (g_2 h)$. Since $g_1 g_1^{-1} = e$, the identity element of $G$, we have $x = e (g_2 h) = g_2 h$. Therefore, $x \in g_2 H$.

Next, let $y \in g_2 H$. This means that there exists $h' \in H$ such that $y = g_2 h'$. We want to show that $y \in g_1 H$. Since $g_1^{-1} g_2 \in H$, we can rewrite $y$ as $y = (g_1 g_1^{-1}) (g_2 h')$. By associativity, this becomes $y = g_1 (g_1^{-1} g_2 h')$. Since $g_1^{-1} g_2 h' \in H$ (since $H$ is a subgroup), we have $y = g_1 h''$ for some $h'' \in H$. Therefore, $y \in g_1 H$.

Since every element in $g_1 H$ is also in $g_2 H$, and vice versa, we can conclude that $g_1 H = g_2 H$.

\probskip

\item[{\bf 6.14}] Let $G$ be a group and suppose $g\in G$, $n> 0$,