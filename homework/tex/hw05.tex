Certainly! Here are the answers to the homework questions:

\begin{enumerate}

\item[{\bf 6.}]
The order of every element in the symmetry group of the square, $D_4$, is as follows:
\begin{itemize}
\item The identity element has order 1.
\item The four rotations (0, 90, 180, and 270 degrees) have order 4.
\item The four reflections (horizontal, vertical, and two diagonal) have order 2.
\end{itemize}

\item[{\bf 12.}]
A cyclic group with exactly one generator is the trivial group $\{e\}$, where $e$ is the identity element. 

A cyclic group with exactly two generators is the group of integers under addition, $\mathbb{Z}$.

A cyclic group with exactly four generators is the group of integers modulo 4, $\mathbb{Z}_4$.

For any positive integer $n$, the group of integers modulo $n$, $\mathbb{Z}_n$, is a cyclic group with exactly $n$ generators.

\item[{\bf 13.}]
For $n \leq 20$, the groups $U(n)$ are cyclic if and only if $n$ is 1, 2, 4, a power of an odd prime, or twice a power of an odd prime. 

Conjecture: The group $U(n)$ is cyclic if and only if $n$ is 1, 2, 4, a power of an odd prime, or twice a power of an odd prime.

This conjecture can be proven using the properties of Euler's totient function and the prime factorization of $n$.

\item[{\bf 28.}]
If $a$ is an element of a group, a generator for the subgroup $\langle a^m \rangle \cap \langle a^n \rangle$ is $a^{\gcd(m,n)}$.

\item[{\bf 30.}]
Suppose $a$ and $b$ are elements of a group. If $|a| = m$ and $|b| = n$ with $\gcd(m,n) = 1$, then $\langle a \rangle \cap \langle b \rangle = \{ e \}$, where $e$ is the identity element.

\item[{\bf 35.}]
The subgroups of $\mathbb{Z}$ are exactly $n\mathbb{Z}$ for $n = 0, 1, 2, \ldots$. In other words, the subgroups of $\mathbb{Z}$ are the sets of integers that are multiples of a fixed integer $n$.

\item[{\bf 38.}]
The order of an element in a cyclic group must divide the order of the group. In other words, if $G$ is a cyclic group of order $n$ and $a$ is an element of $G$ with order $m$, then $m$ divides $n$.

\item[{\bf 47.}]
The given Hasse diagram cannot be the entire subgroup lattice of a group because there is an element in $G$ that does not belong to $J$. In a subgroup lattice, every element in $G$ must belong to at least one subgroup, but in this diagram, there is an element in $G$ that is not in $J$. Therefore, this diagram does not satisfy the requirements of a subgroup lattice.

\end{enumerate}

Let me know if there's anything else I can help you with!