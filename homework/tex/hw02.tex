% Example LaTeX document for GP111 - note % sign indicates a comment
\documentclass[12pt,reqno]{amsart}
\usepackage[top=1.5cm, left=1.5cm,right=1.5cm,bottom=1.5cm]{geometry}
\renewcommand{\baselinestretch}{1.2}
\usepackage{amsmath}
\usepackage{amssymb}
\usepackage{color,hyperref,enumerate,multicol}
\definecolor{darkblue}{rgb}{0.0,0.0,0.3}
\hypersetup{colorlinks,breaklinks,
            linkcolor=darkblue,urlcolor=darkblue,
            anchorcolor=darkblue,citecolor=darkblue}
            
\usepackage{algorithm}
\usepackage{algorithmic}
\pagestyle{empty}
\newcommand{\N}{\ensuremath{\mathbb{N}}}
\newcommand{\Z}{\ensuremath{\mathbb{Z}}}
\newcommand{\R}{\ensuremath{\mathbb{R}}}
\newcommand{\meet}{\ensuremath{\wedge}}
\newcommand{\Meet}{\ensuremath{\bigwedge}}
\newcommand{\join}{\ensuremath{\vee}}
\renewcommand{\emptyset}{\ensuremath{\varnothing}}
\renewcommand{\subset}{\ensuremath{\subsetneq}}
\newcommand{\boldemph}{\emph}
\newcommand{\lcm}{\operatorname{lcm}}

\begin{document}
\thispagestyle{empty}

\noindent \textbf{Math 3140}  \hfill {\bf Homework 2}
\vskip1cm
\noindent {\bf Chapter 2:}  14, 22, 24, 25, 26, 28.  \\
{\bf Due date:} Wednesday, 9/12

\medskip

\begin{enumerate}

%% 14 %%%%%%%%%%%%%%%%%%%%%%%%%%%%%%%%%%%%%%%%%%%%%%%%
\item[{\bf 14.}]
Show that the Principle of Well-Ordering for the natural numbers implies that 0 is 
the smallest natural number.  Use this result to show that the Principle of 
Well-Ordering implies the Principle of Mathematical Induction; that is, show 
that if $S \subseteq {\mathbb N}$ and if $S$ satisfies the two conditions,
\begin{enumerate}
  \item $0 \in S$,
  \item $n \in S$ implies $n + 1 \in S$,
\end{enumerate}
then $S = \mathbb N$.  

\textbf{Answer:} Let's assume that the Principle of Well-Ordering holds for the natural numbers. We want to show that 0 is the smallest natural number. Suppose, for contradiction, that there exists a natural number $n$ such that $n < 0$. But this contradicts the Well-Ordering Principle since there is no natural number less than 0. Therefore, 0 must be the smallest natural number.

Now, let $S \subseteq \mathbb{N}$ be a set that satisfies the two conditions: (1) $0 \in S$, and (2) $n \in S$ implies $n + 1 \in S$. We want to show that $S = \mathbb{N}$. Suppose, for contradiction, that $S \neq \mathbb{N}$. Then there must exist a natural number $m$ such that $m \notin S$. By the Well-Ordering Principle, there must be a smallest natural number $k$ such that $k \notin S$. Since $0 \in S$, $k$ must be greater than 0. But this means that $k - 1$ is a natural number and $k - 1 \in S$ (by the second condition). This contradicts the assumption that $k$ is the smallest natural number not in $S$. Therefore, $S$ must be equal to $\mathbb{N}$.

\bigskip

%%%% %%%%%%%%%%%%%%%%%%%%%%%%%%%%%%%%%%%%%%%%%%%%%%%%
\item[{\bf 22.}]
Let $n \in {\mathbb N}$.  Use the division algorithm to prove that every integer is congruent mod $n$ to precisely one of the integers $0, 1, \ldots, n-1$.  Conclude that if $r$ is an integer, then there is exactly one $s$ in ${\mathbb Z}$ such that $0 \leq s < n$ and $[r] = [s]$.   Hence, the integers are indeed partitioned by congruence mod $n$. 

\textbf{Answer:} Let $a$ be an integer. By the division algorithm, there exist unique integers $q$ and $r$ such that $a = qn + r$ and $0 \leq r < n$. This means that $a$ is congruent to $r$ modulo $n$, i.e., $[a] = [r]$. Since $0 \leq r < n$, $r$ must be one of the integers $0, 1, \ldots, n-1$. Therefore, every integer is congruent modulo $n$ to precisely one of the integers $0, 1, \ldots, n-1$.

Now, let $r$ be an integer. We have shown that $r$ is congruent modulo $n$ to precisely one of the integers $0, 1, \ldots, n-1$. Let $s$ be the integer such that $0 \leq s < n$ and $[r] = [s]$. Since $[r] = [s]$, this means that $r$ and $s$ have the same remainder when divided by $n$. Therefore, $r$ and $s$ are congruent modulo $n$. Hence, the integers are indeed partitioned by congruence modulo $n$.

\bigskip

%%%% %%%%%%%%%%%%%%%%%%%%%%%%%%%%%%%%%%%%%%%%%%%%%%%%
\item[{\bf 24.}]
If $d= \gcd(a, b)$ and $m = \lcm(a, b)$, prove that $dm = |ab|$.

\textbf{Answer:} Let $d = \gcd(a, b)$ and $m = \lcm(a, b)$. By definition, $d$ is the largest positive integer that divides both $a$ and $b$, and $m$ is the smallest positive integer that is divisible by both $a$ and $b$.

Since $d$ divides $a$ and $b$, we can write $a = dx$ and $b = dy$ for some integers $x$ and $y$. Therefore, $ab = (dx)(dy) = d^2(xy)$. Since $d$ is a positive integer, $d^2$ is also a positive integer. Thus, $d^2(xy)$ is positive. Therefore, $|ab| = d^2(xy)$.

Since $m$ is divisible by both $a$ and $b$, we can write $m = az = bz'$ for some integers $z$ and $z'$. Therefore, $m = (dx)z = (dy)z'$. Dividing both sides by $d$, we get $x z = y z'$. Since $x$, $y$, $z$, and $z'$ are integers, $xy$ is divisible by $z$. Let $k$ be the integer such that $xy = kz$. Then we have $d^2(xy) = d^2(kz) = (dk)(dz)$. Since $dk$ and $dz$ are integers, $(dk)(dz)$ is divisible by $d$. Therefore, $d^2(xy)$ is divisible by $d$. 

Since $d^2(xy)$ is divisible by $d$, we can write $d^2(xy) = dm$ for some integer $m$. Dividing both sides by $d$, we get $dm = |ab|$. Therefore, $dm = |ab|$.

\bigskip

%%%% %%%%%%%%%%%%%%%%%%%%%%%%%%%%%%%%%%%%%%%%%%%%%%%%
\item[{\bf 25.}]
Show that $\lcm(a,b) = |ab|$ if and only if $\gcd(a,b) = 1$.

\textbf{Answer:} Let $a$ and $b$ be integers.

First, suppose that $\lcm(a,b) = |ab|$. We want to show that $\gcd(a,b) = 1$. Let $d = \gcd(a,b)$. By definition, $d$ is the largest positive integer that divides both $a$ and $b$. Since $d$ divides $a$ and $b$, we can write $a = dx$ and $b = dy$ for some integers $x$ and $y$. 

We know that $\lcm(a,b) = |ab|$. Substituting $a = dx$ and $b = dy$, we get $\lcm(dx, dy) = |(dx)(dy)|$. Simplifying, we have $\lcm(dx, dy) = d^2|xy|$. Since $\lcm(dx, dy)$ is divisible by $d$, we can write $\lcm(dx, dy) = dk$ for some integer $k$. Dividing both sides by $d$, we get $k = d|xy|$. 

Since $k$ is an integer, $d$ divides $k$. Therefore, $d$ divides $d|xy|$. But $d|xy|$ is equal to $\lcm(dx, dy)$. Since $d$ is the largest positive integer that divides both $a$ and $b$, $d$ must also divide $\lcm(a,b)$. But we know that $\lcm(a,b) = |ab|$. Therefore, $d$ divides $|ab|$. 

Since $d$ divides $|ab|$ and $d$ divides $a$, $d$ must also divide $|ab| - a = b$. Therefore, $d$ is a common divisor of $a$ and $b$. But $d$ is the largest positive integer that divides both $a$ and $b$, so $d$ must be equal to $\gcd(a,b)$. Therefore, $\gcd(a,b) = d$. 

Since $d$ divides $|ab|$ and $d$ divides $a$, $d$ must also divide $|ab| - a = b$. Therefore, $d$ is a common divisor of $a$ and $b$. But $d$ is the largest positive integer that divides both $a$ and $b$, so $d$ must be equal to $\gcd(a,b)$. Therefore, $\gcd(a,b) = d$. 

Therefore, if $\lcm(a,b) = |ab|$, then $\gcd(a,b) = 1$.

Now, suppose that $\gcd(a,b) = 1$. We want to show that $\lcm(a,b) = |ab|$. Let $m = \lcm(a,b)$. By definition, $m$ is the smallest positive integer that is divisible by both $a$ and $b$. 

Since $\gcd(a,b) = 1$, there exist integers $x$ and $y$ such that $ax + by = 1$. Multiplying both sides by $|ab|$, we get $a(|ab|x) + b(|ab|y) = |ab|$. This shows that $|ab|$ is divisible by both $a$ and $b$. 

Now, let $n$ be a positive integer that is divisible by both $a$ and $b$. We want to show that $n$ is greater than or equal to $|ab|$. Since $n$ is divisible by $a$ and $b$, we can write $n = az = bz'$ for some integers $z$ and $z'$. Dividing both sides by $|ab|$, we get $\frac{n}{|ab|} = \frac{az}{|ab|} = \frac{bz'}{|ab|}$. Simplifying, we have $\frac{n}{|ab|} = \frac{z}{b} = \frac{z'}{a}$. Since $\frac{z}{b}$ and $\frac{z'}{a}$ are integers, $\frac{n}{|ab|}$ must also be an integer. 

Since $\frac{n}{|ab|}$ is an integer, $n$ is divisible by $|ab|$. But $|ab|$ is the smallest positive integer that is divisible by both $a$ and $b$. Therefore, $n$ must be greater than or equal to $|ab|$. 

Since $|ab|$ is divisible by both $a$ and $b$, and any positive integer divisible by both $a$ and $b$ is greater than or equal to $|ab|$, $|ab|$ is the smallest positive integer that is divisible by both $a$ and $b$. Therefore, $|ab|$ is equal to $\lcm(a,b)$. 

Therefore, if $\gcd(a,b) = 1$, then $\lcm(a,b) = |ab|$.

Hence, we have shown that