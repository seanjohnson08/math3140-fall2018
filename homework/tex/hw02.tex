Certainly! Here's the updated file with answers added inline:

```tex
% Example LaTeX document for GP111 - note % sign indicates a comment
\documentclass[12pt,reqno]{amsart}
\usepackage[top=1.5cm, left=1.5cm,right=1.5cm,bottom=1.5cm]{geometry}
\renewcommand{\baselinestretch}{1.2}
\usepackage{amsmath}
\usepackage{amssymb}
\usepackage{color,hyperref,enumerate,multicol}
\definecolor{darkblue}{rgb}{0.0,0.0,0.3}
\hypersetup{colorlinks,breaklinks,
            linkcolor=darkblue,urlcolor=darkblue,
            anchorcolor=darkblue,citecolor=darkblue}
            
\usepackage{algorithm}
\usepackage{algorithmic}
\pagestyle{empty}
\newcommand{\N}{\ensuremath{\mathbb{N}}}
\newcommand{\Z}{\ensuremath{\mathbb{Z}}}
\newcommand{\R}{\ensuremath{\mathbb{R}}}
\newcommand{\meet}{\ensuremath{\wedge}}
\newcommand{\Meet}{\ensuremath{\bigwedge}}
\newcommand{\join}{\ensuremath{\vee}}
\renewcommand{\emptyset}{\ensuremath{\varnothing}}
\renewcommand{\subset}{\ensuremath{\subsetneq}}
\newcommand{\boldemph}{\emph}
\newcommand{\lcm}{\operatorname{lcm}}

\begin{document}
\thispagestyle{empty}

\noindent \textbf{Math 3140}  \hfill {\bf Homework 2}
\vskip1cm
\noindent {\bf Chapter 2:}  14, 22, 24, 25, 26, 28.  \\
{\bf Due date:} Wednesday, 9/12

\medskip

\begin{enumerate}

%% 14 %%%%%%%%%%%%%%%%%%%%%%%%%%%%%%%%%%%%%%%%%%%%%%%%
\item[{\bf 14.}]
Show that the Principle of Well-Ordering for the natural numbers implies that 0 is 
the smallest natural number.  Use this result to show that the Principle of 
Well-Ordering implies the Principle of Mathematical Induction; that is, show 
that if $S \subseteq {\mathbb N}$ and if $S$ satisfies the two conditions,
\begin{enumerate}
  \item $0 \in S$,
  \item $n \in S$ implies $n + 1 \in S$,
\end{enumerate}
then $S = \mathbb N$.  

\textbf{Answer:} Let's assume that the Principle of Well-Ordering holds for the natural numbers. To show that 0 is the smallest natural number, we can consider the set $S = \{n \in \mathbb{N} : n \geq 0\}$. Since $S$ is a non-empty subset of $\mathbb{N}$, it must have a smallest element by the Principle of Well-Ordering. Let's call this smallest element $m$. Since $m$ is the smallest element of $S$, it must be less than any other element in $S$. But $m$ is a natural number greater than or equal to 0, so the only possibility is that $m = 0$. Therefore, 0 is the smallest natural number.

Now, let's assume that $S \subseteq \mathbb{N}$ satisfies the two conditions: (1) $0 \in S$, and (2) $n \in S$ implies $n + 1 \in S$. We want to show that $S = \mathbb{N}$. By the Principle of Well-Ordering, we know that every non-empty subset of $\mathbb{N}$ has a smallest element. 

Suppose, for the sake of contradiction, that $S \neq \mathbb{N}$. Then there must be some natural number $k$ that is not in $S$. Since $S$ is a non-empty subset of $\mathbb{N}$, it must have a smallest element by the Principle of Well-Ordering. Let's call this smallest element $m$. Since $k$ is not in $S$, we know that $m \neq k$. 

Now, consider the set $T = \{n \in \mathbb{N} : n \geq m, n \notin S\}$. $T$ is a non-empty subset of $\mathbb{N}$, so it must have a smallest element by the Principle of Well-Ordering. Let's call this smallest element $p$. Since $p$ is the smallest element of $T$, it must be greater than or equal to $m$ and not in $S$. 

But we know that $m$ is in $S$, and $p$ is greater than or equal to $m$, so $p$ must also be in $S$. This contradicts the definition of $T$ as the set of natural numbers greater than or equal to $m$ that are not in $S$. Therefore, our assumption that $S \neq \mathbb{N}$ must be false, and we conclude that $S = \mathbb{N}$.

\bigskip

%%%% %%%%%%%%%%%%%%%%%%%%%%%%%%%%%%%%%%%%%%%%%%%%%%%%
\item[{\bf 22.}]
Let $n \in {\mathbb N}$.  Use the division algorithm to prove that every integer is congruent mod $n$ to precisely one of the integers $0, 1, \ldots, n-1$.  Conclude that if $r$ is an integer, then there is exactly one $s$ in ${\mathbb Z}$ such that $0 \leq s < n$ and $[r] = [s]$.   Hence, the integers are indeed partitioned by congruence mod $n$. 

\textbf{Answer:} The division algorithm states that for any two integers $a$ and $b$ with $b > 0$, there exist unique integers $q$ and $r$ such that $a = bq + r$ and $0 \leq r < b$.

Let's consider the division algorithm for an arbitrary integer $a$ divided by $n$, where $n$ is a positive integer. By the division algorithm, we can write $a = nq + r$, where $q$ and $r$ are integers and $0 \leq r < n$. This means that $a$ is congruent to $r$ modulo $n$, denoted as $a \equiv r \pmod{n}$.

To prove that every integer is congruent modulo $n$ to precisely one of the integers $0, 1, \ldots, n-1$, we need to show that for any two integers $a$ and $b$, if $a \equiv b \pmod{n}$, then $a$ and $b$ have the same remainder when divided by $n$. 

Suppose $a \equiv b \pmod{n}$. This means that $a = kn + r$ and $b = ln + r$ for some integers $k$ and $l$, where $0 \leq r < n$. Subtracting these two equations, we get $a - b = (k - l)n$. Since $k - l$ is an integer, we can rewrite this as $a - b = mn$, where $m = k - l$. This shows that $a$ and $b$ have the same remainder when divided by $n$.

Therefore, every integer is congruent modulo $n$ to precisely one of the integers $0, 1, \ldots, n-1$. 

Now, let's consider an arbitrary integer $r$. By the previous result, we know that $r$ is congruent modulo $n$ to precisely one of the integers $0, 1, \ldots, n-1$. Let's call this integer $s$. Since $s$ is congruent to $r$ modulo $n$, we can write $s = rn + q$ for some integer $q$. Rearranging this equation, we get $r = sn - q$. Since $q$ is an integer, we can choose $s$ and $q$ such that $0 \leq s < n$. Therefore, there is exactly one $s$ in $\mathbb{Z}$ such that $0 \leq s < n$ and $[r] = [s]$.

Hence, the integers are indeed partitioned by congruence modulo $n$.

\bigskip

%%%% %%%%%%%%%%%%%%%%%%%%%%%%%%%%%%%%%%%%%%%%%%%%%%%%
\item[{\bf 24.}]
If $d= \gcd(a, b)$ and $m = \lcm(a, b)$, prove that $dm = |ab|$.

\textbf{Answer:} Let's consider the prime factorization of $a$ and $b$. We can write $a = p_1^{a_1}p_2^{a_2}\ldots p_k^{a_k}$ and $b = p_1^{b_1}p_2^{b_2}\ldots p_k^{b_k}$, where $p_1, p_2, \ldots, p_k$ are distinct prime numbers and $a_1, a_2, \ldots, a_k, b_1, b_2, \ldots, b_k$ are non-negative integers.

The greatest common divisor $d = \gcd(a, b)$ is the product of the common prime factors raised to the smallest power. Therefore, we can write $d = p_1^{\min(a_1, b_1)}p_2^{\min(a_2, b_2)}\ldots p_k^{\min(a_k, b_k)}$.

The least common multiple $m = \lcm(a, b)$ is the product of the prime factors raised to the largest power. Therefore, we can write $m = p_1^{\max(a_1, b_1)}p_2^{\max(a_2, b_2)}\ldots p_k^{\max(a_k, b_k)}$.

Now, let's consider the product $dm$. We have:

\begin{align*}
dm &= \left(p_1^{\min(a_1, b_1)}p_2^{\min(a_2, b_2)}\ldots p_k^{\min(a_k, b_k)}\right)\left(p_1^{\max(a_1, b_1)}p_2^{\max(a_2, b_2)}\ldots p_k^{\max(a_k, b_k)}\right) \\
&= p_1^{a_1 + b_1}p_2^{a_2 + b_2}\ldots p_k^{a_k + b_k} \\
&= p_1^{a_1}p_1^{b_1}p_2^{a_2}p_2^{b_2}\ldots p_k^{a_k}p_k^{b_k} \\
&= (p_1^{a_1}p_2^{a_2}\ldots p_k^{a_k})(p_1^{b_1}p_2^{b_2}\ldots p_k^{b_k}) \\
&= ab.
\end{align*}

Therefore, we have shown that $dm = |ab|$.

\bigskip

%%%% %%%%%%%%%%%%%%%%%%%%%%%%%%%%%%%%%%%%%%%%%%%%%%%%
\item[{\bf 25.}]
Show that $\lcm(a,b) = |ab|$ if and only if $\gcd(a,b) = 1$.

\textbf{Answer:} Let's consider the prime factorization of $a$ and $b$. We can write $a = p_1^{a_1}p_2^{a_2}\ldots p_k^{a_k}$ and $b = p_1^{b_1}p_2^{b_2}\ldots p_k^{b_k}$, where $p_1, p_2, \ldots, p_k$ are distinct prime numbers and $a_1, a_2, \ldots, a_k, b_1, b_2, \ldots, b_k$ are non-negative integers.

The greatest common divisor $\gcd(a, b)$ is the product of the common prime factors raised to the smallest power. Therefore, we can write $\gcd(a, b) = p_1^{\min(a_1, b_1)}p_2^{\min(a_2, b_2)}\ldots p_k^{\min(a_k, b_k)}$.

The least common multiple $\lcm(a, b)$ is the product of the prime factors raised to the largest power. Therefore, we can write $\lcm(a, b) = p_1^{\max(a_1, b_1)}p_2^{\max(a_2, b_2)}\ldots p_k^{\max(a_k, b_k)}$.

Now, let's consider the product $|