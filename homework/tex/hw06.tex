\documentclass[12pt,reqno]{amsart}
\usepackage[top=2cm, left=2cm,right=2cm,bottom=2cm]{geometry}
\renewcommand{\baselinestretch}{1.2}
\usepackage{amsmath}
\usepackage{amssymb}
\usepackage{color,hyperref,enumerate,multicol}
\definecolor{darkblue}{rgb}{0.0,0.0,0.3}
\hypersetup{colorlinks,breaklinks,
            linkcolor=darkblue,urlcolor=darkblue,
            anchorcolor=darkblue,citecolor=darkblue}
            
\usepackage{algorithm}
\usepackage{algorithmic}
\pagestyle{empty}
\newcommand{\N}{\ensuremath{\mathbb{N}}}
\newcommand{\Z}{\ensuremath{\mathbb{Z}}}
\newcommand{\R}{\ensuremath{\mathbb{R}}}
\newcommand{\meet}{\ensuremath{\wedge}}
\newcommand{\Meet}{\ensuremath{\bigwedge}}
\newcommand{\join}{\ensuremath{\vee}}
\renewcommand{\emptyset}{\ensuremath{\varnothing}}
\renewcommand{\subset}{\ensuremath{\subsetneq}}
\newcommand{\boldemph}{\emph}
\newcommand{\lcm}{\operatorname{lcm}}

\newcommand{\probskip}{\vskip1cm}

\begin{document}
\thispagestyle{empty}

\noindent \textbf{Math 3140} \hfill {\bf Homework 6}\\[4pt]
\noindent {\bf Chapter 5:}   1bd, 3bd, 4, 6, 17, 18, 27.\hfill {\bf Due date:} Friday, 10/05\\  
   Additional suggested exercises: 29, 31, 32, 33.  \\


\medskip


\begin{enumerate}[{\bf 1.}]

%% 1 %%%%%%%%%%%%%%%%%%%%%%%%%%%%%%%%%%%%%%%%%%%%%%%%
\item %1
Write the following permutations in cycle notation.
\begin{multicols}{2}
\begin{enumerate}
 
\item
\[
\begin{pmatrix}
1 & 2 & 3 & 4 & 5 \\
2 & 4 & 1 & 5 & 3
\end{pmatrix}
=(13254)
\]

\item
\[
\begin{pmatrix}
1 & 2 & 3 & 4 & 5 \\
4 & 2 & 5 & 1 & 3
\end{pmatrix}
=(14253)
\]

\item
\[
\begin{pmatrix}
1 & 2 & 3 & 4 & 5 \\
3 & 5 & 1 & 4 & 2
\end{pmatrix}
=(13524)
\]

\item
\[
\begin{pmatrix}
1 & 2 & 3 & 4 & 5 \\
1 & 4 & 3 & 2 & 5
\end{pmatrix}
=(14)(23)
\]

\end{enumerate}
\end{multicols}

\probskip
 
%% 3 %%%%%%%%%%%%%%%%%%%%%%%%%%%%%%%%%%%%%%%%%%%%%%%%
\item[{\bf 3.}] 
Express the following permutations as products of transpositions and
identify them as even or odd.  (Only (b) and (d) are required.)
\begin{multicols}{2}
\begin{enumerate}
 
\item
$(14356)=(14)(13)(15)(16)$, odd

 \item
$(156)(234)=(15)(16)(23)(24)$, even
 
 \item
$(1426)(142)=(16)(14)(12)$, even
 
 \item
$(17254)(1423)(154632)=(17)(12)(14)(15)(23)(24)(36)$, even
 
 \item
$(142637)=(17)(12)(14)(13)(16)$, odd
 
\end{enumerate}
\end{multicols}

\probskip
 
%% 4 %%%%%%%%%%%%%%%%%%%%%%%%%%%%%%%%%%%%%%%%%%%%%%%%
\item[{\bf 4.}] 
Find $(a_1, a_2, \ldots, a_n)^{-1}$.

$(a_1, a_2, \ldots, a_n)^{-1}=(a_n, a_{n-1}, \ldots, a_2, a_1)$

\probskip
 
%% 6 %%%%%%%%%%%%%%%%%%%%%%%%%%%%%%%%%%%%%%%%%%%%%%%%
\item[{\bf 6.}] 
Find all of the subgroups in $A_4$. What is the order of each
subgroup? 

The subgroups of $A_4$ are $\{(), (12)(34), (13)(24), (14)(23)\}$, each of order 2, and $\{(), (123), (132), (124), (142), (134), (143), (234), (243)\}$, each of order 3.

\probskip

%% 17 %%%%%%%%%%%%%%%%%%%%%%%%%%%%%%%%%%%%%%%%%%%%%%%%
\item[{\bf 17.}] 
Prove that $S_n$ is nonabelian for $n \geq 3$.

To prove that $S_n$ is nonabelian for $n \geq 3$, we need to show that there exist two permutations $\alpha$ and $\beta$ in $S_n$ such that $\alpha\beta \neq \beta\alpha$. Let $\alpha=(12)$ and $\beta=(23)$. Then $\alpha\beta=(12)(23)=(123)$ and $\beta\alpha=(23)(12)=(132)$. Since $(123) \neq (132)$, we have shown that $S_n$ is nonabelian for $n \geq 3$.

\probskip
 
%% 18 %%%%%%%%%%%%%%%%%%%%%%%%%%%%%%%%%%%%%%%%%%%%%%%%
\item[{\bf 18.}] 
Prove that $A_n$ is nonabelian for $n \geq 4$.

To prove that $A_n$ is nonabelian for $n \geq 4$, we need to show that there exist two permutations $\alpha$ and $\beta$ in $A_n$ such that $\alpha\beta \neq \beta\alpha$. Let $\alpha=(123)$ and $\beta=(124)$. Then $\alpha\beta=(123)(124)=(134)$ and $\beta\alpha=(124)(123)=(143)$. Since $(134) \neq (143)$, we have shown that $A_n$ is nonabelian for $n \geq 4$.

\probskip
 
%% 27 %%%%%%%%%%%%%%%%%%%%%%%%%%%%%%%%%%%%%%%%%%%%%%%%
\item[{\bf 27.}] 
Let $G$ be a group and define a map $\lambda_g : G \rightarrow G$ by
$\lambda_g(a) = g a$.  Prove that $\lambda_g$ is a permutation of $G$.

To prove that $\lambda_g$ is a permutation of $G$, we need to show that $\lambda_g$ is a bijection. 

\textbf{Injectivity:} Suppose $\lambda_g(a_1) = \lambda_g(a_2)$. Then $ga_1 = ga_2$. Left multiplying both sides by $g^{-1}$, we get $a_1 = a_2$. Therefore, $\lambda_g$ is injective.

\textbf{Surjectivity:} Let $b \in G$. We want to show that there exists $a \in G$ such that $\lambda_g(a) = b$. Let $a = g^{-1}b$. Then $\lambda_g(a) = g(g^{-1}b) = (gg^{-1})b = b$. Therefore, $\lambda_g$ is surjective.

Since $\lambda_g$ is both injective and surjective, it is a bijection. Hence, $\lambda_g$ is a permutation of $G$.

\newpage

%   Additional suggested exercises: 29, 31, 32, 33.  \\
\noindent {\bf Additional suggested exercises:} 29, 31, 32, 33.  
\\

\item[{\bf 29.}]  % 29
Recall that the \boldemph{center}\index{Group!center of} of a group $G$ is
\[
Z(G) = \{ g \in G : \mbox{$gx = xg$ for all $x \in G$} \}.
\]
Find the center of $D_8$. What about the center of $D_{10}$? What is
the center of $D_n$? 

The center of $D_8$ is $\{(), (12)(34)\}$.

The center of $D_{10}$ is $\{(), (12)(34)(56)(78), (135)(246), (153)(264), (14)(23)(56)(78), (15)(26)(37)(48), (16)(25)(38)(47), (17)(28)(35)(46), (18)(27)(36)(45)\}$.

The center of $D_n$ for $n \geq 3$ is $\{(), (12)(34)\}$.

\probskip
 
 
\item[{\bf 31.}]  % 31
For $\alpha$ and $\beta$ in $S_n$, define $\alpha \sim \beta$ if there
exists an $\sigma \in S_n$ such that 
$\sigma \alpha \sigma^{-1} = \beta$.  Show that $\sim$ is an equivalence
relation on $S_n$. 

To show that $\sim$ is an equivalence relation on $S_n$, we need to show that it satisfies the three properties of reflexivity, symmetry, and transitivity.

\textbf{Reflexivity:} For any $\alpha \in S_n$, we have $\alpha \sim \alpha$ since we can choose $\sigma = \text{id}$.

\textbf{Symmetry:} If $\alpha \sim \beta$, then there exists $\sigma \in S_n$ such that $\sigma \alpha \sigma^{-1} = \beta$. Taking the inverse of both sides, we have $(\sigma \