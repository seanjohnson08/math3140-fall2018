Certainly! Here is the updated file with answers added:

\documentclass[12pt,reqno]{amsart}
\usepackage[top=2cm, left=2cm,right=2cm,bottom=2cm]{geometry}
\renewcommand{\baselinestretch}{1.2}
\usepackage{amsmath}
\usepackage{amssymb}
\usepackage{color,hyperref,enumerate,multicol}
\definecolor{darkblue}{rgb}{0.0,0.0,0.3}
\hypersetup{colorlinks,breaklinks,
            linkcolor=darkblue,urlcolor=darkblue,
            anchorcolor=darkblue,citecolor=darkblue}
            
\usepackage{algorithm}
\usepackage{algorithmic}
\pagestyle{empty}
\newcommand{\N}{\ensuremath{\mathbb{N}}}
\newcommand{\Z}{\ensuremath{\mathbb{Z}}}
\newcommand{\R}{\ensuremath{\mathbb{R}}}
\newcommand{\meet}{\ensuremath{\wedge}}
\newcommand{\Meet}{\ensuremath{\bigwedge}}
\newcommand{\join}{\ensuremath{\vee}}
\renewcommand{\emptyset}{\ensuremath{\varnothing}}
\renewcommand{\subset}{\ensuremath{\subsetneq}}
\newcommand{\boldemph}{\emph}
\newcommand{\lcm}{\operatorname{lcm}}

\newcommand{\probskip}{\vskip1cm}

\begin{document}
\thispagestyle{empty}

\noindent \textbf{Math 3140} \hfill {\bf Homework 6}\\[4pt]
\noindent {\bf Chapter 5:}   1bd, 3bd, 4, 6, 17, 18, 27.\hfill {\bf Due date:} Friday, 10/05\\  
   Additional suggested exercises: 29, 31, 32, 33.  \\


\medskip


\begin{enumerate}[{\bf 1.}]

%% 1 %%%%%%%%%%%%%%%%%%%%%%%%%%%%%%%%%%%%%%%%%%%%%%%%
\item %1
Write the following permutations in cycle notation.
\begin{multicols}{2}
\begin{enumerate}
 
\item
\[
\begin{pmatrix}
1 & 2 & 3 & 4 & 5 \\
2 & 4 & 1 & 5 & 3
\end{pmatrix}
\]
Answer: $(13254)$

\item
\[
\begin{pmatrix}
1 & 2 & 3 & 4 & 5 \\
4 & 2 & 5 & 1 & 3
\end{pmatrix}
\]
Answer: $(14253)$

\item
\[
\begin{pmatrix}
1 & 2 & 3 & 4 & 5 \\
3 & 5 & 1 & 4 & 2
\end{pmatrix}
\]
Answer: $(13254)$

\item
\[
\begin{pmatrix}
1 & 2 & 3 & 4 & 5 \\
1 & 4 & 3 & 2 & 5
\end{pmatrix}
\]
Answer: $(14)(23)$

\end{enumerate}
\end{multicols}

\probskip
 
%% 3 %%%%%%%%%%%%%%%%%%%%%%%%%%%%%%%%%%%%%%%%%%%%%%%%
\item[{\bf 3.}] 
Express the following permutations as products of transpositions and
identify them as even or odd.  (Only (b) and (d) are required.)
\begin{multicols}{2}
\begin{enumerate}
 
\item
$(14356)$
Answer: $(14)(13)(15)(16)$, odd

 \item
$(156)(234)$
Answer: $(15)(16)(24)$, even
 
 \item
$(1426)(142)$
Answer: $(16)(24)$, even
 
 \item
$(17254)(1423)(154632)$
Answer: $(14)(16)(23)(25)(34)$, even
 
 \item
$(142637)$
Answer: $(16)(24)(27)(36)$, odd
 
\end{enumerate}
\end{multicols}

\probskip
 
%% 4 %%%%%%%%%%%%%%%%%%%%%%%%%%%%%%%%%%%%%%%%%%%%%%%%
\item[{\bf 4.}] 
Find $(a_1, a_2, \ldots, a_n)^{-1}$.
Answer: $(a_1, a_2, \ldots, a_n)$

\probskip
 
%% 6 %%%%%%%%%%%%%%%%%%%%%%%%%%%%%%%%%%%%%%%%%%%%%%%%
\item[{\bf 6.}] 
Find all of the subgroups in $A_4$. What is the order of each
subgroup? 
Answer: The subgroups of $A_4$ are $\{(), (12)(34), (13)(24), (14)(23)\}$, each with order 2, and $\{(), (123), (132), (124), (142), (134), (143), (234), (243)\}$, each with order 3.

\probskip

%% 17 %%%%%%%%%%%%%%%%%%%%%%%%%%%%%%%%%%%%%%%%%%%%%%%%
\item[{\bf 17.}] 
Prove that $S_n$ is nonabelian for $n \geq 3$.
Answer: To prove that $S_n$ is nonabelian, we can find two permutations $\alpha$ and $\beta$ such that $\alpha\beta \neq \beta\alpha$. For example, let $\alpha = (12)$ and $\beta = (23)$. Then $\alpha\beta = (13)$ and $\beta\alpha = (12)$, so $\alpha\beta \neq \beta\alpha$. Therefore, $S_n$ is nonabelian for $n \geq 3$.

\probskip
 
%% 18 %%%%%%%%%%%%%%%%%%%%%%%%%%%%%%%%%%%%%%%%%%%%%%%%
\item[{\bf 18.}] 
Prove that $A_n$ is nonabelian for $n \geq 4$.
Answer: Similar to the previous question, we can find two permutations $\alpha$ and $\beta$ in $A_n$ such that $\alpha\beta \neq \beta\alpha$. For example, let $\alpha = (123)$ and $\beta = (124)$. Then $\alpha\beta = (134)$ and $\beta\alpha = (243)$, so $\alpha\beta \neq \beta\alpha$. Therefore, $A_n$ is nonabelian for $n \geq 4$.

\probskip
 
%% 27 %%%%%%%%%%%%%%%%%%%%%%%%%%%%%%%%%%%%%%%%%%%%%%%%
\item[{\bf 27.}] 
Let $G$ be a group and define a map $\lambda_g : G \rightarrow G$ by
$\lambda_g(a) = g a$.  Prove that $\lambda_g$ is a permutation of $G$.
Answer: To prove that $\lambda_g$ is a permutation of $G$, we need to show that it is a bijection. 
First, we will show that $\lambda_g$ is injective. Suppose $\lambda_g(a) = \lambda_g(b)$. Then $ga = gb$, and multiplying both sides by $g^{-1}$ on the left gives $a = b$. Therefore, $\lambda_g$ is injective.
Next, we will show that $\lambda_g$ is surjective. Let $c \in G$. Since $G$ is a group, there exists an element $d \in G$ such that $gd = c$. Then $\lambda_g(d) = gd = c$, so $\lambda_g$ is surjective.
Since $\lambda_g$ is both injective and surjective, it is a bijection. Therefore, $\lambda_g$ is a permutation of $G$.

\newpage

%   Additional suggested exercises: 29, 31, 32, 33.  \\
\noindent {\bf Additional suggested exercises:} 29, 31, 32, 33.  
\\

\item[{\bf 29.}]  % 29
Recall that the \boldemph{center}\index{Group!center of} of a group $G$ is
\[
Z(G) = \{ g \in G : \mbox{$gx = xg$ for all $x \in G$} \}.
\]
Find the center of $D_8$. What about the center of $D_{10}$? What is
the center of $D_n$? 
Answer: The center of $D_8$ is $\{(), (12)(34)\}$, the center of $D_{10}$ is $\{(), (12)(34)(56)(78)\}$, and the center of $D_n$ for $n \geq 3$ is $\{(), (12)(34)\}$.

\probskip
 
 
\item[{\bf 31.}]  % 31
For $\alpha$ and $\beta$ in $S_n$, define $\alpha \sim \beta$ if there
exists an $\sigma \in S_n$ such that 
$\sigma \alpha \sigma^{-1} = \beta$.  Show that $\sim$ is an equivalence
relation on $S_n$. 
Answer: To show that $\sim$ is an equivalence relation on $S_n$, we need to show that it is reflexive, symmetric, and transitive.
- Reflexive: For any $\alpha \in S_n$, we can choose $\sigma = \text{id}$, the identity permutation. Then $\sigma \alpha \sigma^{-1} = \alpha$, so $\alpha \sim \alpha$. Therefore, $\sim$ is reflexive.
- Symmetric: Suppose $\alpha \sim \beta$. Then there exists $\sigma \in S_n$ such that $\sigma \alpha \sigma^{-1} = \beta$. Taking the inverse of both sides gives $(\sigma^{-1}) \beta (\sigma^{-1})^{-1} = \alpha$, so $\beta \sim \alpha$. Therefore, $\sim$ is symmetric.
- Transitive: Suppose $\alpha \sim \beta$ and $\beta \sim \gamma$. Then there exist $\sigma_1, \sigma_2 \in S_n$ such that $\sigma_1 \alpha \sigma_1^{-1} =