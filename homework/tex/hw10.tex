Certainly! Here are the updated contents of the file with the answers added inline:

\documentclass[12pt,reqno]{amsart}
\usepackage[top=2cm, left=2cm,right=2cm,bottom=2cm]{geometry}
\renewcommand{\baselinestretch}{1.2}
\usepackage{amsmath}
\usepackage{amssymb}
\usepackage{scalefnt}
\usepackage{tikz}
\usepackage{color,hyperref,enumerate,multicol}
\definecolor{darkblue}{rgb}{0.0,0.0,0.3}
\hypersetup{colorlinks,breaklinks,
            linkcolor=darkblue,urlcolor=darkblue,
            anchorcolor=darkblue,citecolor=darkblue}
            
\usepackage{algorithm}
\usepackage{algorithmic}
\pagestyle{empty}
\newcommand{\N}{\ensuremath{\mathbb{N}}}
\newcommand{\Z}{\ensuremath{\mathbb{Z}}}
\newcommand{\R}{\ensuremath{\mathbb{R}}}
\newcommand{\bL}{\ensuremath{\mathbf{L}}}
\newcommand{\bP}{\ensuremath{\mathbf{P}}}
\newcommand{\bQ}{\ensuremath{\mathbf{Q}}}
\newcommand{\bA}{\ensuremath{\mathbf{A}}}
\newcommand{\bB}{\ensuremath{\mathbf{B}}}
\newcommand{\bG}{\ensuremath{\mathbf{G}}}
\newcommand{\bH}{\ensuremath{\mathbf{H}}}
\newcommand{\invG}{\ensuremath{\operatorname{inv}^{\bG}}}
\newcommand{\invH}{\ensuremath{\operatorname{inv}^{\bH}}}
\newcommand{\meet}{\ensuremath{\wedge}}
\newcommand{\Meet}{\ensuremath{\bigwedge}}
\newcommand{\<}{\ensuremath{\langle}}
\renewcommand{\>}{\ensuremath{\rangle}}
\newcommand{\join}{\ensuremath{\vee}}
\renewcommand{\emptyset}{\ensuremath{\varnothing}}
\renewcommand{\subset}{\ensuremath{\subsetneq}}
\newcommand{\boldemph}{\emph}
\newcommand{\lcm}{\ensuremath{\operatorname{lcm}}}
\newcommand{\Sym}{\ensuremath{\operatorname{Sym}}}
%\newcommand{\bG}{\ensuremath{\mathbf{G}}}

\newcommand{\probskip}{\vskip5mm}
\usepackage{xspace}
\newcommand{\subject}{MATH\xspace}
\newcommand{\coursenumber}{3140\xspace}
\newcommand{\semester}{Fall 2018\xspace}

\begin{document}
\thispagestyle{empty}

\noindent \textbf{\subject \coursenumber Homework 10}

\smallskip

\noindent {\bf Exercises.} 1 below plus Judson Ch 10. 1abe, 5, 10, 11, 13acd, 14.\\
{\bf Due date:} Friday, 11/09


\medskip

\begin{enumerate}
\item[{\bf 1.}] Let $\bG = \<G, \cdot, ^{-1}, e\>$ be a finite group of order $n$.  
Take the set $G$ (the elements of $\bG$) and consider the group of all
permutations of these elements.  This group is sometimes denoted by $\Sym(G)$;
note that it is isomorphic to the symmetric group $S_n$ of permutations of
an $n$-element set.
Now fix an element $a\in G$ and recall that the function
$\lambda_a: G \rightarrow G$, defined by $\lambda_a(g) = a\cdot g$, is a
permutation of the set $G$.  That is, $\lambda_a$ belongs to the
permutation group $\Sym(G)$.
\begin{enumerate}[(a)]
\item 
Prove that the function $\lambda: G \rightarrow \Sym(G)$ is a group
homomorphism.  
\medskip
\textbf{Answer:} To prove that $\lambda: G \rightarrow \Sym(G)$ is a group homomorphism, we need to show that for any $x, y \in G$, $\lambda(xy) = \lambda(x)\lambda(y)$. Let $x, y \in G$. Then we have:
\[\lambda(xy) = a \cdot (xy) = (a \cdot x)(a \cdot y) = \lambda(x)\lambda(y).\]
Therefore, $\lambda$ is a group homomorphism.
\medskip
\item What is the kernel of $\lambda$?\footnote{Recall that the kernel of a function $f: X \rightarrow Y$ is the subset of
  $X\times X$ defined by 
\[
\ker f = \{(x_1,x_2) : f(x_1) = f(x_2)\}.
\]
As you have already proved, the kernel is an equivalence relation on $X$.}
\medskip
\textbf{Answer:} The kernel of $\lambda$ is the set of all elements $x \in G$ such that $\lambda(x) = e$, where $e$ is the identity element of $G$. In other words, the kernel of $\lambda$ is the set $\{x \in G : a \cdot x = e\}$.
\medskip
\item Let $N$ denote the equivalence class of $\ker\lambda$ that contains the
  identity element $e$ of $G$.  Prove that $N$ is a normal subgroup of $G$.
\end{enumerate}
\textbf{Answer:} To prove that $N$ is a normal subgroup of $G$, we need to show that for any $g \in G$ and $n \in N$, $gng^{-1} \in N$. Let $g \in G$ and $n \in N$. Since $N$ is an equivalence class of $\ker\lambda$, we have $n \in \ker\lambda$, which means $\lambda(n) = e$. Now consider $\lambda(gng^{-1})$. We have:
\[\lambda(gng^{-1}) = a \cdot (gng^{-1}) = (a \cdot g)(a \cdot n)(a \cdot g^{-1}) = \lambda(g)\lambda(n)\lambda(g^{-1}) = \lambda(g)e\lambda(g^{-1}) = \lambda(g)\lambda(g^{-1}) = \lambda(gg^{-1}) = \lambda(e) = e.\]
Therefore, $gng^{-1} \in \ker\lambda$, which implies $gng^{-1} \in N$. Hence, $N$ is a normal subgroup of $G$.

\probskip

%% 1 %%%%%%%%%%%%%%%%%%%%%%%%%%%%%%%%%%%%%%%%%%%%%%%%
\item[{\bf 10.1}] For each of the following groups $G$, determine whether $H$ is a normal
subgroup of $G$. If $H$ is a normal subgroup, write out a Cayley table
for the factor group $G/H$.
\begin{enumerate}
\item[(a)]
$G = S_4$ and $H = A_4$
\textbf{Answer:} $H$ is a normal subgroup of $G$. The Cayley table for the factor group $G/H$ is as follows:

\[
\begin{array}{c|cccc}
\cdot & H & (12)H & (13)H & (23)H \\
\hline
H & H & (12)H & (13)H & (23)H \\
(12)H & (12)H & H & (23)H & (13)H \\
(13)H & (13)H & (23)H & H & (12)H \\
(23)H & (23)H & (13)H & (12)H & H \\
\end{array}
\]

\item[(b)]
$G = A_5$ and $H = \{ (1), (123), (132) \}$
\textbf{Answer:} $H$ is a normal subgroup of $G$. The Cayley table for the factor group $G/H$ is as follows:

\[
\begin{array}{c|ccc}
\cdot & H & (12)H & (12)(34)H \\
\hline
H & H & (12)H & (12)(34)H \\
(12)H & (12)H & H & (12)(34)H \\
(12)(34)H & (12)(34)H & (12)(34)H & H \\
\end{array}
\]

\item[(e)]
$G = {\mathbb Z}$ and $H = 5 {\mathbb Z}$
\textbf{Answer:} $H$ is a normal subgroup of $G$. The Cayley table for the factor group $G/H$ is as follows:

\[
\begin{array}{c|ccccc}
+ & H & 1+H & 2+H & 3+H & 4+H \\
\hline
H & H & 1+H & 2+H & 3+H & 4+H \\
1+H & 1+H & 2+H & 3+H & 4+H & H \\
2+H & 2+H & 3+H & 4+H & H & 1+H \\
3+H & 3+H & 4+H & H & 1+H & 2+H \\
4+H &