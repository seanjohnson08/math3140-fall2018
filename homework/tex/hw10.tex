Certainly! Here are the answers to the homework questions:

1. (a) To prove that the function $\lambda: G \rightarrow \Sym(G)$ is a group homomorphism, we need to show that for any $g_1, g_2 \in G$, $\lambda(g_1 \cdot g_2) = \lambda(g_1) \circ \lambda(g_2)$, where $\circ$ denotes function composition in $\Sym(G)$.

Let $g_1, g_2 \in G$. Then, $\lambda(g_1 \cdot g_2) = (g_1 \cdot g_2) \cdot g = g_1 \cdot (g_2 \cdot g) = \lambda(g_1) \circ \lambda(g_2)$, where the first equality follows from the definition of $\lambda$, the second equality follows from the associativity of the group operation, and the third equality follows from the definition of $\lambda$.

Therefore, $\lambda$ is a group homomorphism.

(b) The kernel of $\lambda$ is the set of elements $g \in G$ such that $\lambda(g) = \lambda(e)$, where $e$ is the identity element of $G$. In other words, the kernel of $\lambda$ is the set of elements $g \in G$ such that $g \cdot a = a$ for all $a \in G$.

Since $e \cdot a = a$ for all $a \in G$, we have $e \in \ker \lambda$. Conversely, if $g \in \ker \lambda$, then $g \cdot a = a$ for all $a \in G$. Setting $a = e$, we get $g \cdot e = e$, which implies $g = e$. Therefore, $\ker \lambda = \{e\}$.

(c) Let $N$ denote the equivalence class of $\ker \lambda$ that contains the identity element $e$ of $G$. To prove that $N$ is a normal subgroup of $G$, we need to show that for any $g \in G$ and $n \in N$, $g \cdot n \cdot g^{-1} \in N$.

Let $g \in G$ and $n \in N$. Since $n \in N$, we have $n \cdot a = a$ for all $a \in G$. Now, consider $g \cdot n \cdot g^{-1} \cdot a$. We can rewrite this expression as $(g \cdot n \cdot g^{-1}) \cdot (g \cdot a)$. Since $n \cdot a = a$ for all $a \in G$, we have $(g \cdot n \cdot g^{-1}) \cdot (g \cdot a) = g \cdot (n \cdot a) = g \cdot a$. Therefore, $g \cdot n \cdot g^{-1} \in N$.

Hence, $N$ is a normal subgroup of $G$.

10.1:
(a) $H = A_4$ is a normal subgroup of $G = S_4$. The factor group $G/H$ is isomorphic to the cyclic group of order 2.

(b) $H = \{(1), (123), (132)\}$ is not a normal subgroup of $G = A_5$.

(e) $H = 5\mathbb{Z}$ is a normal subgroup of $G = \mathbb{Z}$. The factor group $G/H$ is isomorphic to the cyclic group of order 5.

10.5: The intersection of two normal subgroups is a normal subgroup. Let $N_1$ and $N_2$ be normal subgroups of a group $G$. To show that $N_1 \cap N_2$ is a normal subgroup, we need to show that for any $g \in G$ and $n \in N_1 \cap N_2$, $g \cdot n \cdot g^{-1} \in N_1 \cap N_2$.

Since $n \in N_1 \cap N_2$, we have $n \in N_1$ and $n \in N_2$. Since $N_1$ and $N_2$ are normal subgroups, we have $g \cdot n \cdot g^{-1} \in N_1$ and $g \cdot n \cdot g^{-1} \in N_2$. Therefore, $g \cdot n \cdot g^{-1} \in N_1 \cap N_2$.

Hence, $N_1 \cap N_2$ is a normal subgroup of $G$.

10.10: Let $H$ be a subgroup of index 2 of a group $G$. To prove that $H$ is a normal subgroup of $G$, we need to show that for any $g \in G$, $g \cdot H = H \cdot g$.

Since $H$ has index 2 in $G$, there are exactly two left cosets and two right cosets of $H$ in $G$. Let $H$ and $gH$ be the left cosets of $H$ in $G$, and let $H$ and $Hg$ be the right cosets of $H$ in $G$. Since there are only two left cosets and two right cosets, we must have $gH = H$ and $Hg = H$.

Therefore, $g \cdot H = H \cdot g$, and hence $H$ is a normal subgroup of $G$.

10.11: If a group $G$ has exactly one subgroup $H$ of order $k$, we can prove that $H$ is normal in $G$ by showing that for any $g \in G$, $g \cdot H = H \cdot g$.

Let $g \in G$. Since $H$ is a subgroup of $G, g \cdot H$ is a subset of $G$. Since $H$ is the only subgroup of order $k$, $g \cdot H$ must also have order $k$. Therefore, $g \cdot H = H$.

Similarly, $H \cdot g$ is a subset of $G$ and has order $k$. Therefore, $H \cdot g = H$.

Hence, $g \cdot H = H \cdot g$, and therefore $H$ is a normal subgroup of $G$.

10.13:
(a) The center of $S_3$ is the set $\{e\}$, where $e$ is the identity element of $S_3$. This is because in $S_3$, the only element that commutes with all other elements is the identity element.

(c) The center of any group $G$ is a normal subgroup of $G$. To prove this, we need to show that for any $g \in G$ and $z \in Z(G)$, $g \cdot z \cdot g^{-1} \in Z(G)$.

Let $g \in G$ and $z \in Z(G)$. Since $z \in Z(G)$, we have $z \cdot a = a \cdot z$ for all $a \in G$. Now, consider $g \cdot z \cdot g^{-1} \cdot a$. We can rewrite this expression as $(g \cdot z \cdot g^{-1}) \cdot (g \cdot a)$. Since $z \cdot a = a \cdot z$ for all $a \in G$, we have $(g \cdot z \cdot g^{-1}) \cdot (g \cdot a) = g \cdot (z \cdot a) = g \cdot (a \cdot z) = (g \cdot a) \cdot z = (g \cdot a) \cdot (g \cdot z)$. Therefore, $g \cdot z \cdot g^{-1} \cdot a = (g \cdot a) \cdot (g \cdot z)$.

Since $g \cdot a$ and $g \cdot z$ are both elements of $G$, we have $g \cdot z \cdot g^{-1} \in Z(G)$.

Hence, the center of $G$ is a normal subgroup of $G$.

(d) If $G / Z(G)$ is cyclic, then $G$ is abelian. To prove this, let $N$ be the factor group $G / Z(G)$. Since $N$ is cyclic, there exists an element $x \in G$ such that for each $a \in G$, there exists $m \in \mathbb{N}$ such that $aZ = x^mZ$, where $Z$ is the equivalence class of $Z(G)$ that contains the identity element of $G$.

Fix $a, b \in G$. We want to show that $ab = ba$. Since $aZ = x^mZ$ and $bZ = x^nZ$ for some $m$ and $n$, we have $a \cdot x^m = x^m \cdot a$ and $b \cdot x^n = x^n \cdot b$. Multiplying these equations, we get $(a \cdot x^m) \cdot (b \cdot x^n) = (x^m \cdot a) \cdot (x^n \cdot b)$. Simplifying, we have $a \cdot (x^m \cdot b) \cdot x^n = (x^m \cdot a) \cdot (b \cdot x^n)$. Since $x^m \cdot b$ and $x^n \cdot a$ are both elements of $G$, we have $a \cdot (x^m \cdot b) \cdot x^n = (x^m \cdot a) \cdot (b \cdot x^n)$. Canceling $x^n$ from both sides, we get $a \cdot (x^m \cdot b) = (x^m